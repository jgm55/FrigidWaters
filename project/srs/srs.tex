\documentclass{article}

\usepackage[margin=1in]{geometry}
\usepackage{parskip}
\usepackage{datetime}
\usepackage{pgfgantt}
\usepackage{url}
\usepackage{hyperref}
\usepackage[utf8]{inputenc}
\newdateformat{required}{\twodigit{\THEMONTH}/\twodigit{\THEDAY}/\THEYEAR}
\raggedbottom
\begin{document}

\begin{titlepage}
    \begin{center}
        \begin{huge}
        House in Your Head \\[1cm]
        Team G1-FrigidWaters \\[2.2cm]
        { \bfseries System Requirements Specification } \\[1cm]
        Cycle \# 1\\[2.2cm]
        Date: \required\today\\[1cm]
        \end{huge}
    \end{center}
    \null \vfill
    \begin{large}
        Team Members: \\[0.5cm]
        Name: Samuel Bever\\[0.5cm]
        Name: Michael Conway\\[0.5cm]
        Name: Joseph Muoio\\[0.5cm]
        Name: Kyle Patron\\[0.5cm]
        Name: Kevin Zakszewski
    \end{large}
\end{titlepage}
\section*{\centering Table of Contents}
\makeatletter
\@starttoc{toc}
\newcommand{\hsubsubsection}{
\@startsection{subsubsection}{3}{\z@}%
                                     {-3.25ex\@plus -1ex \@minus -.2ex}%
                                     {-1.5ex \@plus -.2ex}% Formerly 1.5ex \@plus .2ex
                                     {R\normalfont\normalsize}}
\newcommand{\hparagraph}{
\@startsection{paragraph}{4}{\z@}%
                                     {-3.25ex\@plus -1ex \@minus -.2ex}%
                                     {-1.5ex \@plus -.2ex}% Formerly 1.5ex \@plus .2ex
                                     {R\normalfont\normalsize}}
\newcommand{\hsubparagraph}{
\@startsection{subparagraph}{5}{\z@}%
                                     {-3.25ex\@plus -1ex \@minus -.2ex}%
                                     {-1.5ex \@plus -.2ex}% Formerly 1.5ex \@plus .2ex
                                     {R\normalfont\normalsize}}
\setcounter{secnumdepth}{5}
\makeatother
\newpage
 

{\color{red}xxxxxxxxxxxxxxxxxxxxxxxxxxxxTODO: Remove names from sectionsxxxxxxxx}
\section{Introduction}
{\color{red}xxxxxxxxxxxxxxxxxxxxxxxxxxxxTODO: Who is doing this?xxxxxxxxxxxxxxxxx}

[See IEEE 830 Section 5.1 for reference.]

\subsection{Purpose}
The purpose of this document is to describe the implementation of the House in Your Head system that 
was described in the House in Your Head launch report document. In the proceeding sections, you will 
find the scope, necessary language and definitions, overview of the project, and system requirements 
defined.

\subsection{Scope}

This system is intended as a way for user's to make basic state changes in automated home setting
using a Brain Computer Interface. Basic state changes are characterized, generally, by having a binary 
setting (on or off). This system also includes a user interface that is controlled by binary actions. Complex 
systems with multiple or intermediate states are out of the scope of the Brain Computer Interface portion 
this project. Some settings that have multiple states can be maintained by a administrator. This 
administrator will need to do so via a mouse and keyboard. 

The potential users of this system will be people who suffer from ALS as well as administrators 
who will be assisting the general users set up the program and software. Dr. Sara Feldman, The ALS 
Center of Hope at Drexel University, and Professor Jeff Salvage are the primary stakeholders and 
sponsors of this project.

\subsection{Definitions, Acronyms, and Abbreviations}
\begin{description}
    \item[EEG] Can refer to:
        \begin{itemize}
            \item Electroencephalography - Recording of the brain's electrical
                activity 
	        \item Electroencephalogram - The device that is used to record the
	            brain's electrical activity
        \end{itemize}
    \item[Emotiv] The electroencephalogram hardware device, created by Emotiv
        Limited, used to read the user's brain activity (EEG)
    \item[Brain Computer Interface (BCI)] The class of devices that the Emotiv
        belongs to
    \item[Amyotrophic Lateral Sclerosis (ALS)] A neurodegenerative disorder
        that our target users suffer from. The main characteristics of ALS
        that we are concerned with in the scope of this project are the
        limited movement and mobility to complete paralysis.
\end{description}

\subsection{References}
[If this report is citing information from other documents such as the prior reports, list the references here. 
For example, the S report may make reference to the launch report or the system design specification 
report may reference to the system requirement specification report.]

\subsection{Overview}
[The section is describing the rest of the SRS document, providing a roadmap of what are going to be 
covered in the next few sections.]

Over the next few sections, this document will cover the overall product description and the 
specific system requirements. The product description consists of specific product data such including 
interfaces, constraints, functions, user profile, and requirements and dependencies. The specific system 
requirements include external interfaces, required functions, performance requirements, database 
requirements, design constraints, and software system attributes.

\newpage

\section{Overall Description}

\subsection{Product Perspective}

\subsubsection{System Interface}

[Describe the information that is passed to and from other external systems. Detailed descriptions 
will be presented in Section 3.1.1 System Interfaces. A context diagram is good to illustrate the 
information flow between your system and the external entities and datastores.]

The user wears an Emotiv headset which sends information about their thoughts to the connected computer. From here, the information is interpreted and sent to the home automation controller. The home automation controller controls the home components such as the television and lights. 

\subsubsection{User Interface}

	The user interface consists of the Emotiv device which is attached to the user's head and reads their brain's electrical impulses. These are converted into commands that are then used as input. The exact input will be one of two states – neutral or active. The user will use the Emotiv as input to interact with the graphical user interface.

\subsubsection{Hardware Interfaces}

The Emotive device transmits the EEG to the computer through Bluetooth to a receiver which attaches to a Windows computer's USB port.

\subsubsection{Software Interfaces}

The Emotiv framework provides a set of functions which allows access 	to the EEG data.

\subsubsection{Communication Interfaces}

\subsubsection{Memory Constraints}

This system is supposed to run on consumer systems, so it cannot use more than 500 MB of RAM.

\subsubsection{Operations}

\subsubsection{Site Adaption Requirements}

[If your system is going to be used in multiple physical locations, what changes are needed for the system to adapt to another location?]
It must be able to run in both a home and a hospital setup. To do this, it will be made to run on standard Windows PCs.

\subsection{Product Functions - kyle}
\begin{itemize}
\item Gather EEG data from the user
\item Analyse and filter the EEG data to make a reliable signal
\item Provide user with a menu from which to select home automation actions
\item Perform selected home automation actions
\end{itemize}

\subsection{User Characteristics - Kevin}

Two categories of users are considered:

\begin{description}
    \item[Patients] are those suffering from ALS who are the primary users of
        the system. The severity of the disorder in each patient can range
        from limited mobility to full paralysis; the system assumes the
        latter.
    \item[Caretakers] are individuals not suffering from ALS who are
        responsible for Patients. They are assumed to be present during system
        setup and administrative tasks, but \textbf{not} during other use
        cases.
\end{description}

\subsubsection{Use Cases}

% Template
%\begin{description}
%    \item[Scenario Name:]
%    \item[ID number:]
%    \item[Description:]
%    \item[Trigger:]
%    \item[Type:]
%    \item[Major Inputs:] \hfill \\
%        \begin{tabular}{l l}
%            \textbf{Description} & \textbf{Source} \\
%        \end{tabular}
%    \item[Major Outputs:] \hfill \\
%        \begin{tabular}{l l}
%            \textbf{Description} & \textbf{Destination} \\
%        \end{tabular}
%    \item[Major Steps Performed:] \hfill
%        \begin{enumerate}
%        \end{enumerate}
%\end{description}

\begin{description}
    \item[Scenario Name:] Calibrate for Patient
    \item[ID number:] UC005
    \item[Description:] Caretaker sets up and calibrates the system for a Patient.
    \item[Trigger:] Caretaker puts the device on a Patient, starts the
        software and makes the menu selection to calibrate the device.
    \item[Type:] External
    \item[Major Inputs:] \hfill \\
        \begin{tabular}{l l}
            \textbf{Description} & \textbf{Source} \\
            EEG data & Patient (via device) \\
            Supporting data & Patient (via Caretaker) \\
        \end{tabular}
    \item[Major Outputs:] \hfill \\
        \begin{tabular}{l l}
            \textbf{Description} & \textbf{Destination} \\
            Instructions and menus & Patient, Caretaker \\
        \end{tabular}
    \item[Major Steps Performed:] \hfill
        \begin{enumerate}
            \item Begin calibration routine.
            \item Confirm that device is functioning correctly.
            \item Guide Patient and Caretaker through calibration steps.
            \item Save calibration data and display confirmation message.
        \end{enumerate}
\end{description}

\hfill \\

\begin{description}
    \item[Scenario Name:] Turn Lights On
    \item[ID number:] UC010
    \item[Description:] Patient turns a light on. (This is a representative
        home automation task; others will work similarly.)
    \item[Trigger:] Patient wants to turn a light on and thinks the activation thought.
    \item[Type:] External
    \item[Major Inputs:] \hfill \\
        \begin{tabular}{l l}
            \textbf{Description} & \textbf{Source} \\
            EEG data & Patient (via device) \\
        \end{tabular}
    \item[Major Outputs:] \hfill \\
        \begin{tabular}{l l}
            \textbf{Description} & \textbf{Destination} \\
            Menu display & Patient \\
            Light activation signal & Home Automation System
        \end{tabular}
    \item[Major Steps Performed:] \hfill
        \begin{enumerate}
            \item Display initial menu on screen.
            \item Accept thought input from user and accept selection when
                appropriate level of certainty is met.
            \item Display next menu on screen and repeat until leaf (turn
                lights on task) is reached.
            \item Display confirmation screen and send signal to turn lights
                on.
            \item Return to idle state.
        \end{enumerate}
\end{description}

\subsection{Constraints - Sam}

\subsection{Assumptions and Dependencies - Sam}

\subsection{Apportioning of Requirements - Kyle}
During the first term, we will focus on getting a reliable signal from the EEG and developing the initial ideas for our user interfaces. For the second term, we will mostly finish those two tasks and begin working on the home automation portion of the project. The final term will consist mostly of finishing the integration with the home automation portion and wrapping up any loose ends.
\newpage

\section{Specific Requirements}
This section contains a detailed description of all of the requirements for the system and its features. 

\subsection{External Interfaces}
This section gives a description of the hardware and software interfaces. Also included is a basic prototype of the UI.

\subsubsection{System Interfaces}

The user will use a device called the Emotive to provide input to the system through their thoguht. The system will interpret the data and communicate with the home automation controller which will control all the aspects of the home automation. Included in this will be turning the lights on and off, working the television, and adjusting the thermostat. This will all happen through the home automation controller interface.

\subsubsection{User Interfaces}
{\color{red}xxxxxxxxxxxxxxxxxxxxTODO: INSERT MOCKUPS AND WRITE THIS SECTION - Joexxxxxxxxx}

\subsection{Functions - Mike}

%Template
%\textbf{SR2XX - NAME}
%\begin{description}
%    \item[Input:]
%    \item[Action:]
%    \item[Output:]
%    \item[Notes:] 
%    \item[Priority:]
%\end{description}

\textbf{SR205 - Calibrate}
\begin{description}
    \item[Input:] EEG data, supporting data from Caretaker
    \item[Action:] Derive calibration parameters for the Patient
    \item[Output:] Calibration parameters (internal; to be saved)
    \item[Notes:] The parameters (including the input thoughts requested from
        the user) should be chosen to give a reliable signal. Recalibration
        should be possible, as Patients' thought patterns may change over
        time.
    \item[Priority:] Must Have
\end{description}

\hfill \\

\textbf{SR210 - Activate menu via EEG}
\begin{description}
    \item[Input:] EEG data
    \item[Action:] Detect activation signal
    \item[Output:] Display menu
    \item[Notes:] Necessary for fully-paralyzed patients; others may have some
        other method of activation. Depends on calibration data.
    \item[Priority:] Must Have
\end{description}

\hfill \\

\textbf{SR215 - Perform Home Automation Action}
\begin{description}
    \item[Input:] EEG data
    \item[Action:] Interpret signal as menu selections
    \item[Output:] Signal to HAS corresponding to user request
    \item[Notes:] Depends on calibration data
    \item[Priority:] Must Have
\end{description}
\addtocontents{toc}{\protect\setcounter{tocdepth}{2}}
\subsection{Performance Requirements - Kyle}
\textbf{R3300} Must run well on a standard computer

\setlength\parindent{24pt}
\textbf{R3301} Must utilize less than 500 MB of RAM

\textbf{R3302} Must utilize less than 5 GB of disk

\textbf{R3303} Must run smoothly with an 3.2 GHz Intel i3 processor

\setlength\parindent{0pt}

\textbf{R3310} Must support one and only one terminal

\textbf{R3320} Must support one and only one user at a time

\textbf{R3330} User interaction must be smooth
\setlength\parindent{24pt} 

\textbf{R3333} 90\% of time, selection must take less than 5 seconds

\textbf{R3334} A trained user must be able to select 10 options within 2 minutes
\setlength\parindent{0pt}

\subsection{Logical Database Requirements - Kyle}
\textbf{R3400} No databases shall be used


\subsection{Design Constraints}
{\color{red}xxxxxxxxxxxxxxxxxxxxxxxxxxxxTODO: Who is doing this?xxxxxxxxxxxxxxxxxxxxxxxxxx}

\subsubsection{Standards Compliance}

\subsection{Software Systems Attributes - Kyle}

\subsubsection{Reliability} 
\textbf{R3610} Our system must fail to detect a selection no more than 10\% of the time

\textbf{R3611} Our system must unintentionally select no more than once every 30 minutes

\subsubsection{Availability}
\textbf{R3620} Our system must be able to run for 24 hours without restarting 99\% of the time

\subsubsection{Security}
\textbf{R3630} Our system must have no network communication

\textbf{R3631} Our system must not log client interactions

\subsubsection{Maintainability}
\textbf{R3640} The menu selections must be defined by a configuration file

\textbf{R3641} New home automation features must be able to be added through a configuration file

\subsubsection{Portability}
\textbf{R3650} Our system will only run on Windows 7 and Windows 8 machines 

\textbf{R3651} There shall be no machine specific code

\textbf{R3652} There must only be one version of the code that can run on all machines

\addtocontents{toc}{\protect\setcounter{tocdepth}{3}}
%TODO: fix bottom

\newpage
{\color{red}xxxxxxxxxxxxxxxxxxxxxxxxxxxxTODO: FIX THE CONTRIBUTIONSxxxxxxxxxxxxxxxxxxxxxxxxxx}
\section*{\centering Table of Contributions}
\begin{tabular}{| l | l | l | l |}
    \hline
     & Section & Writing & Editing \\
    \hline \hline
    1 & Project & Joe Muoio  & Mike Conway \\ \hline
    2 & Team & Everyone & Sam Bever \\ \hline
    3 & Plan & Kyle Patron & Kevin Zakszewski \\ \hline
\end{tabular}
\newpage
\noindent I certify that:
\begin{itemize}
\item This paper/project/exam is entirely my own work.
\item I have not quoted the words of any other person from a printed source or a website without indicating what has been quoted and providing an appropriate citation.
\item I have not submitted this paper / project to satisfy the requirements of any other course.
\end{itemize}

\vspace{1cm}
\noindent\makebox[\textwidth][l]{
Signature:
\makebox[5cm][l] {\underline{Samuel Bever}} 
\ \ Date:
\makebox[4cm][l] {\underline{\required\today}} 
}


\vspace{0.5cm}
\noindent\makebox[\textwidth][l]{
Signature:
\makebox[5cm][l] {\underline{Michael Conway}} 
\ \ Date:
\makebox[4cm][l] {\underline{\required\today}} 
}

\vspace{0.5cm}
\noindent\makebox[\textwidth][l]{
Signature:
\makebox[5cm][l] {\underline{Joe Muoio}} 
\ \ Date:
\makebox[4cm][l] {\underline{\required\today}} 
}

\vspace{0.5cm}
\noindent\makebox[\textwidth][l]{
Signature:
\makebox[5cm][l] {\underline{Kyle Patron}} 
\ \ Date:
\makebox[4cm][l] {\underline{\required\today}} 
}

\vspace{0.5cm}
\noindent\makebox[\textwidth][l]{
Signature:
\makebox[5cm][l] {\underline{Kevin Zakszewski}} 
\ \ Date:
\makebox[4cm][l] {\underline{\required\today}} 
}

\vspace{\fill}
\subsection*{Grading}
The grade is given on the basis of quality, clarity, presentation, completeness, and writing of each section in the report. This is the grade of the group. Individual grades will be assigned at the end of the term when peer reviews are collected.
\end{document}
