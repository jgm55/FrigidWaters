\documentclass{article}

\usepackage[margin=1in]{geometry}
\usepackage{parskip}
\usepackage{datetime}
\usepackage{pgfgantt}
\usepackage{url}
\usepackage{hyperref}
\usepackage[utf8]{inputenc}
\newdateformat{required}{\twodigit{\THEMONTH}/\twodigit{\THEDAY}/\THEYEAR}
\raggedbottom
\begin{document}

\begin{titlepage}
    \begin{center}
        \begin{huge}
        House in Your Head \\[1cm]
        Team G1-FrigidWaters \\[2.2cm]
        { \bfseries System Requirements Specification } \\[1cm]
        Cycle \# 1\\[2.2cm]
        Date: \required\today\\[1cm]
        \end{huge}
    \end{center}
    \null \vfill
    \begin{large}
        Team Members: \\[0.5cm]
        Name: Samuel Bever\\[0.5cm]
        Name: Michael Conway\\[0.5cm]
        Name: Joseph Muoio\\[0.5cm]
        Name: Kyle Patron\\[0.5cm]
        Name: Kevin Zakszewski
    \end{large}
\end{titlepage}
\section*{\centering Table of Contents}
\makeatletter
\@starttoc{toc}
\newcommand{\hsubsubsection}{
\@startsection{subsubsection}{3}{\z@}%
                                     {-3.25ex\@plus -1ex \@minus -.2ex}%
                                     {-1.5ex \@plus -.2ex}% Formerly 1.5ex \@plus .2ex
                                     {R\normalfont\normalsize}}
\newcommand{\hparagraph}{
\@startsection{paragraph}{4}{\z@}%
                                     {-3.25ex\@plus -1ex \@minus -.2ex}%
                                     {-1.5ex \@plus -.2ex}% Formerly 1.5ex \@plus .2ex
                                     {R\normalfont\normalsize}}
\newcommand{\hsubparagraph}{
\@startsection{subparagraph}{5}{\z@}%
                                     {-3.25ex\@plus -1ex \@minus -.2ex}%
                                     {-1.5ex \@plus -.2ex}% Formerly 1.5ex \@plus .2ex
                                     {R\normalfont\normalsize}}
\setcounter{secnumdepth}{5}
\makeatother
\newpage
 

{\color{red}xxxxxxxxxxxxxxxxxxxxxxxxxxxxTODO: Remove names from sectionsxxxxxxxx}
\section{Introduction}

\subsection{Purpose}

The purpose of this document is to describe the requirements of the House in
Your Head system. The remaining sections define the scope of the project,
introduce necessary language and definitions, give an overview of the project,
and lay out the requirements for the system.

\subsection{Scope}

This system is intended as a way for users to make basic state changes in an
automated home setting using a Brain Computer Interface. Basic states to be
changed are characterized by having a binary setting (on or off). This system
also includes a user interface that is controlled by binary actions. Complex
systems with multiple or intermediate states are out of the scope of the Brain
Computer Interface portion of this project. Some settings that have multiple
states can be maintained by a administrator. This administrator will need to
do so via a mouse and keyboard. 

The potential users of this system will be people who suffer from ALS, as well
as the administrators who will be assisting the general users set up the
program and software. Dr. Sara Feldman, The ALS Center of Hope at Drexel
University, and Professor Jeff Salvage are the primary stakeholders and
sponsors of this project.

\subsection{Definitions, Acronyms, and Abbreviations}

\begin{description}
    \item[EEG] Can refer to:
        \begin{itemize}
            \item Electroencephalography - Recording of the brain's electrical
                activity 
	        \item Electroencephalogram - The device that is used to record the
	            brain's electrical activity
        \end{itemize}
    \item[Emotiv] The electroencephalogram hardware device, created by Emotiv
        Limited, used to read the user's brain activity (EEG)
    \item[Brain Computer Interface (BCI)] The class of devices that the Emotiv
        belongs to
    \item[Amyotrophic Lateral Sclerosis (ALS)] A neurodegenerative disorder
        that our target users suffer from. The main characteristics of ALS
        that we are concerned with in the scope of this project are the
        limited movement and mobility to complete paralysis.
\end{description}


\subsection{Overview}

Section 2 discusses the background and context of this system and indicates
general requirements and constraints. Section 3 lists specific requirements
for the system.

\newpage

\section{Overall Description}

\subsection{Product Perspective}

\subsubsection{System Interface}

The user wears an Emotiv headset which sends information about their thoughts to the connected computer. From here, the information is interpreted and sent to the home automation controller. The home automation controller controls the home components such as the television and lights. 

\subsubsection{User Interface}

The user interface consists of the Emotiv device which is attached to the user's head and reads their brain's electrical impulses. These are converted into commands that are then used as input. The exact input will be one of two states – neutral or active. The user will use the Emotiv as input to interact with the graphical user interface.

\subsubsection{Hardware Interfaces}

The Emotive device transmits the EEG to the computer through Bluetooth to a receiver which attaches to a Windows computer's USB port.

\subsubsection{Software Interfaces}

The Emotiv framework provides a set of functions which allows access to the EEG data.

\subsubsection{Communication Interfaces}

The system will have no direct network communication.

\subsubsection{Memory Constraints}

The system will be run on consumer systems, so its RAM footprint must be
limited accordingly.

\subsubsection{Operations}

\subsubsection{Site Adaption Requirements}

The system must be able to run in both a home and a hospital setup. No
modifications will be required to change between these two environments.

\subsection{Product Functions}

\begin{itemize}
\item Gather EEG data from the user
\item Analyse and filter the EEG data to make a reliable signal
\item Provide user with a menu from which to select home automation actions
\item Perform selected home automation actions
\end{itemize}

\subsection{User Characteristics - Kevin}

Two categories of users are considered:

\begin{description}
    \item[Patients] are those suffering from ALS who are the primary users of
        the system. The severity of the disorder in each patient can range
        from limited mobility to full paralysis; the system assumes the
        latter.
    \item[Caretakers] are individuals not suffering from ALS who are
        responsible for Patients. They are assumed to be present during system
        setup and administrative tasks, but \textbf{not} during other use
        cases.
\end{description}

\subsubsection{Use Cases}

% Template
%\begin{description}
%    \item[Scenario Name:]
%    \item[ID number:]
%    \item[Description:]
%    \item[Trigger:]
%    \item[Type:]
%    \item[Major Inputs:] \hfill \\
%        \begin{tabular}{l l}
%            \textbf{Description} & \textbf{Source} \\
%        \end{tabular}
%    \item[Major Outputs:] \hfill \\
%        \begin{tabular}{l l}
%            \textbf{Description} & \textbf{Destination} \\
%        \end{tabular}
%    \item[Major Steps Performed:] \hfill
%        \begin{enumerate}
%        \end{enumerate}
%\end{description}

\begin{description}
    \item[Scenario Name:] Calibrate for Patient
    \item[ID number:] UC005
    \item[Description:] Caretaker sets up and calibrates the system for a Patient.
    \item[Trigger:] Caretaker puts the device on a Patient, starts the
        software and makes the menu selection to calibrate the device.
    \item[Type:] External
    \item[Major Inputs:] \hfill \\
        \begin{tabular}{l l}
            \textbf{Description} & \textbf{Source} \\
            EEG data & Patient (via device) \\
            Supporting data & Patient (via Caretaker) \\
        \end{tabular}
    \item[Major Outputs:] \hfill \\
        \begin{tabular}{l l}
            \textbf{Description} & \textbf{Destination} \\
            Instructions and menus & Patient, Caretaker \\
        \end{tabular}
    \item[Major Steps Performed:] \hfill
        \begin{enumerate}
            \item Begin calibration routine.
            \item Confirm that device is functioning correctly.
            \item Guide Patient and Caretaker through calibration steps.
            \item Save calibration data and display confirmation message.
        \end{enumerate}
\end{description}

\hfill \\

\begin{description}
    \item[Scenario Name:] Turn Lights On
    \item[ID number:] UC010
    \item[Description:] Patient turns a light on. (This is a representative
        home automation task; others will work similarly.)
    \item[Trigger:] Patient wants to turn a light on and thinks the activation thought.
    \item[Type:] External
    \item[Major Inputs:] \hfill \\
        \begin{tabular}{l l}
            \textbf{Description} & \textbf{Source} \\
            EEG data & Patient (via device) \\
        \end{tabular}
    \item[Major Outputs:] \hfill \\
        \begin{tabular}{l l}
            \textbf{Description} & \textbf{Destination} \\
            Menu display & Patient \\
            Light activation signal & Home Automation System
        \end{tabular}
    \item[Major Steps Performed:] \hfill
        \begin{enumerate}
            \item Display initial menu on screen.
            \item Accept thought input from user and accept selection when
                appropriate level of certainty is met.
            \item Display next menu on screen and repeat until leaf (turn
                lights on task) is reached.
            \item Display confirmation screen and send signal to turn lights
                on.
            \item Return to idle state.
        \end{enumerate}
\end{description}

\subsection{Constraints - Sam}

One of the most immediately obvious constraints on the entire system is the
user's muscular degeneration. This limitation is the inspiration for the
system and the prime motivator for this project and must be kept in mind at
all times.

Another constraint is the average user's familiarity with computer systems. An
interface that may work well for an experienced programmer may not work well
for an inexperienced computer user. Designing for the inexperienced user is
necessary.

A final issue is that users may have other medical issues that need to be kept
in mind. Those that correlate with ALS are most important, as are vision
issues, specifically color blindness, as the primary feedback from our system
to the user will be visual. A user that cannot distinguish between different
shades may be completely unable to use key features of the system. For
example, if button toggles are set up as a red-green selector, then a
red-green colorblind user will be unable to use those toggles.

\subsection{Assumptions and Dependencies - Sam}

There are many assumptions contained in the design of the system. First and
foremost is that users must have the ability to read and write English.
While the system can later be adapted to other languages, being unable to
interact with a user interface in English will make the system impossible to
work with.

The Emotiv requires a large amount of concentration, so any user needs to be
able to concentrate to effectively use it. If a user cannot concentrate,
their commands will not be recognized by the program.

A user of the system is assumed to be able to see or hear options in the
program. A user that is unable to see or hear will have no other methods of
interfacing with the program, making it unusable.

Users of the system must have access to all of the equipment required to
power a computer, as well as reliable electricity service. Without this the
device and the computer it interfaces with will not work.

Additionally, the computer that the user owns must meet the minimum system
requirements. Without this assumption, the program may behave unpredictably.

\subsection{Apportioning of Requirements}

During the first term, we will focus on getting a reliable signal from the EEG
and developing the initial ideas for our user interfaces. For the second term,
we will mostly finish those two tasks and begin working on the home automation
portion of the project. The final term will consist of finishing the
integration with the home automation portion and resolving any remaining
issues.

\newpage

\section{Specific Requirements}

\subsection{External Interfaces}

This section gives a description of the hardware and software interfaces. Also included is a basic prototype of the UI.

\subsubsection{System Interfaces}

\textbf{SR105} The system will accept input from the Emotiv device and
distinguish at least two thought-states of the patient.

\textbf{SR110} The system will interface with a home control system to turn
lights on and off upon user request.

\textbf{SR115} The system will interface with a home control system to operate
a television upon user request.

\textbf{SR115} The system will interface with a home control system to adjust
the thermostat upon user request.

\subsubsection{User Interfaces}

{\color{red}xxxxxxxxxxxxxxxxxxxxTODO: INSERT MOCKUPS AND WRITE THIS SECTION - Joexxxxxxxxx}

\subsection{Functions - Mike}

%Template
%\textbf{SR2XX - NAME}
%\begin{description}
%    \item[Input:]
%    \item[Action:]
%    \item[Output:]
%    \item[Notes:] 
%    \item[Priority:]
%\end{description}

\textbf{SR205 - Calibrate}
\begin{description}
    \item[Input:] EEG data, supporting data from Caretaker
    \item[Action:] Derive calibration parameters for the Patient
    \item[Output:] Calibration parameters (internal; to be saved)
    \item[Notes:] The parameters (including the input thoughts requested from
        the user) should be chosen to give a reliable signal. Recalibration
        should be possible, as Patients' thought patterns may change over
        time.
    \item[Priority:] Must Have
\end{description}

\hfill \\

\textbf{SR210 - Activate menu via EEG}
\begin{description}
    \item[Input:] EEG data
    \item[Action:] Detect activation signal
    \item[Output:] Display menu
    \item[Notes:] Necessary for fully-paralyzed patients; others may have some
        other method of activation. Depends on calibration data.
    \item[Priority:] Must Have
\end{description}

\hfill \\

\textbf{SR215 - Perform Home Automation Action}
\begin{description}
    \item[Input:] EEG data
    \item[Action:] Interpret signal as menu selections
    \item[Output:] Signal to HAS corresponding to user request
    \item[Notes:] Depends on calibration data
    \item[Priority:] Must Have
\end{description}
\addtocontents{toc}{\protect\setcounter{tocdepth}{2}}
\subsection{Performance Requirements}

\textbf{SR305} The system must use less than 512 MB of RAM at all times while running.

\textbf{SR310} The system must use less than 5 GB of persistent storage space when installed.

\textbf{SR315} The system must run smoothly on a 3.2 GHz Intel i3 processor.

\textbf{SR320} The system will support one and only one terminal.

\textbf{SR325} The system will support one and only one user at a time.

\textbf{SR330} 90\% of time, menu selection must take less than 5 seconds.

\textbf{SR335} A trained user must be able to select 10 options within 2
minutes.

\subsection{Logical Database Requirements}

\textbf{SR405} The system will store calibration data for one user.

\subsection{Design Constraints}

{\color{red}xxxxxxxxxxxxxxxxxxxxxxxxxxxxTODO: Who is doing this?xxxxxxxxxxxxxxxxxxxxxxxxxx}

{\color{red}xxxxxxxxxxxxxxxxxxxxxxxxxxxxDoes any more need to be done?xxxxxxxxxxxxxxxxxxxxxxxxxx}

\subsubsection{Standards Compliance}

The system does not need to comply with any external standards.

\subsection{Software System Attributes}

\subsubsection{Reliability} 

\textbf{SR612} The system must fail to detect a selection no more than 10\% of
the time.

\textbf{SR615} The system must unintentionally select no more than once every
30 minutes.

\subsubsection{Availability}

\textbf{SR622} The system must be able to run for 24 hours without restarting
99\% of the time.

\subsubsection{Security}

\textbf{SR632} The system must have no network communication.

\textbf{SR635} The system must not log client interactions other than storing
calibration data..

\subsubsection{Maintainability}

\textbf{SR642} The menu selections must be defined by a configuration file.

\textbf{SR645} New home automation features must be able to be added through a
configuration file.

\subsubsection{Portability}

\textbf{SR652} The system will run on all Windows 7 and Windows 8 machines
meeting the minimum requirements.

\addtocontents{toc}{\protect\setcounter{tocdepth}{3}}
%TODO: fix bottom

\newpage
{\color{red}xxxxxxxxxxxxxxxxxxxxxxxxxxxxTODO: FIX THE CONTRIBUTIONSxxxxxxxxxxxxxxxxxxxxxxxxxx}
\section*{\centering Table of Contributions}
\begin{tabular}{| l | l | l | l |}
    \hline
     & Section & Writing & Editing \\
    \hline \hline
    1 & 2.2, 2.6, 3.3, 3.4, 3.6  & Kyle Patron & \\ \hline
\end{tabular}
\newpage
\noindent I certify that:
\begin{itemize}
\item This paper/project/exam is entirely my own work.
\item I have not quoted the words of any other person from a printed source or a website without indicating what has been quoted and providing an appropriate citation.
\item I have not submitted this paper / project to satisfy the requirements of any other course.
\end{itemize}

\vspace{1cm}
\noindent\makebox[\textwidth][l]{
Signature:
\makebox[5cm][l] {\underline{Samuel Bever}} 
\ \ Date:
\makebox[4cm][l] {\underline{\required\today}} 
}


\vspace{0.5cm}
\noindent\makebox[\textwidth][l]{
Signature:
\makebox[5cm][l] {\underline{Michael Conway}} 
\ \ Date:
\makebox[4cm][l] {\underline{\required\today}} 
}

\vspace{0.5cm}
\noindent\makebox[\textwidth][l]{
Signature:
\makebox[5cm][l] {\underline{Joe Muoio}} 
\ \ Date:
\makebox[4cm][l] {\underline{\required\today}} 
}

\vspace{0.5cm}
\noindent\makebox[\textwidth][l]{
Signature:
\makebox[5cm][l] {\underline{Kyle Patron}} 
\ \ Date:
\makebox[4cm][l] {\underline{\required\today}} 
}

\vspace{0.5cm}
\noindent\makebox[\textwidth][l]{
Signature:
\makebox[5cm][l] {\underline{Kevin Zakszewski}} 
\ \ Date:
\makebox[4cm][l] {\underline{\required\today}} 
}

\vspace{\fill}
\subsection*{Grading}
The grade is given on the basis of quality, clarity, presentation, completeness, and writing of each section in the report. This is the grade of the group. Individual grades will be assigned at the end of the term when peer reviews are collected.
\end{document}
