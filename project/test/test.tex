\documentclass{article}

\usepackage[margin=1in]{geometry}
\usepackage{parskip}
\usepackage{datetime}
\usepackage{url}
\usepackage{hyperref}
\usepackage[utf8]{inputenc}
\usepackage{graphicx}

\newdateformat{required}{\twodigit{\THEMONTH}/\twodigit{\THEDAY}/\THEYEAR}
\raggedbottom
\begin{document}

\begin{titlepage}
    \begin{center}
        \begin{huge}
        House in Your Head \\[1cm]
        Team G1-FrigidWaters \\[2.2cm]
        { \bfseries Test Specification } \\[1cm]
        Cycle \# 3\\[2.2cm]
        Date: \required\today\\[1cm]
        \end{huge}
    \end{center}
    \null \vfill
    \begin{large}
        Team Members: \\[0.5cm]
        Name: Samuel Bever\\[0.5cm]
        Name: Michael Conway\\[0.5cm]
        Name: Joseph Muoio\\[0.5cm]
        Name: Kyle Patron\\[0.5cm]
        Name: Kevin Zakszewski
    \end{large}
\end{titlepage}
\section*{\centering Table of Contents}
\makeatletter
\@starttoc{toc}
\newcommand{\hsubsubsection}{
\@startsection{subsubsection}{3}{\z@}%
                                     {-3.25ex\@plus -1ex \@minus -.2ex}%
                                     {-1.5ex \@plus -.2ex}% Formerly 1.5ex \@plus .2ex
                                     {R\normalfont\normalsize}}
\newcommand{\hparagraph}{
\@startsection{paragraph}{4}{\z@}%
                                     {-3.25ex\@plus -1ex \@minus -.2ex}%
                                     {-1.5ex \@plus -.2ex}% Formerly 1.5ex \@plus .2ex
                                     {R\normalfont\normalsize}}
\newcommand{\hsubparagraph}{
\@startsection{subparagraph}{5}{\z@}%
                                     {-3.25ex\@plus -1ex \@minus -.2ex}%
                                     {-1.5ex \@plus -.2ex}% Formerly 1.5ex \@plus .2ex
                                     {R\normalfont\normalsize}}
\setcounter{secnumdepth}{5}
\makeatother
\newpage
 

\section{Introduction}

\subsection{Purpose}

The purpose of this document is to document the test specifications for the House in Your Head Project. This is written for test engineers so that they know which tests to perform to make sure the system is functioning properly. This section is the introduction. The following sections lay out Functional Tests, Usability Tests, Performance Tests, and Compliance Tests.

\subsection{References}

% TODO: Finish this when rest is written
% [If this report is citing information from other documents such as the prior reports, list the references here. For example, the test specification report may make reference to the SRS or the SDS reports.]

This document References the Software Requirements Specifications document in Sections 2 and 4. Please see the SRS for descriptions of the functions of the system and for performance requirements of the system.

\newpage

\section{Functional Test Specifications}
% Every Subsection follows the below example

% Template
%\subsection{Requirement id - Requirement Name}
%\subsubsection{TestID - Test Name}
%\begin{tabular}{| l | l |}
%    \hline
%	Objective & [Define what is going to be tested.  Use the same terminology as you define the functions.] \\ \hline
%	Conditions and Procedures & [Define the conditions that are needed for the test.  E.g. state of the system, user actions, content of data files, values of data elements, input values etc.] \\ \hline
%	Expected Results & [Define the expected result when the test is executed.  E.g. output values, system responses, event triggering, etc.] \\ \hline
%	Actual Results & [This field provides a place for the test engineers to record the results of the test.] \\ \hline
%\end{tabular}

For each of the below tests, the term ``strong thought'' refers to the accept thought for the patient. This will vary by what works best for the patient. This could be thinking a strong emotional thought such as a thought that evokes feelings of anger or sadness, or it could be the thought of moving a body part such as an arm or leg.

\subsection{SR205 - Calibrate}
\subsubsection{T205-01 - Assisted Calibration}
\begin{tabular}{| l | p{12cm} |}
    \hline
	Requirement & Deriving calibration parameters for a Patient \\ \hline
	Conditions & A Patient and a Caretaker are present and in possession of an
	Emotiv device and a computer with the software installed. The Patient is
	wearing the Emotiv device. \\ \hline
    Procedure &
\begin{enumerate}
    \item The Caretaker initiates the calibration procedure.
    \item The Patient responds to prompts from the graphical interface to
        think a neutral thought, until the system indicates that it is
        finished. After this, the user is instructed to think a strong thought.
\end{enumerate} \\ \hline
	Expected Results & The screen displays a message indicating calibration
has completed successfully, and then returns to the idle state. \\ \hline
Actual Results & \vspace{1cm} \\ \hline
\end{tabular}

\hfill \\

\subsection{SR210 - Activate menu via EEG}
\subsubsection{T210-01 - Activate menu via EEG}
\begin{tabular}{| l | p{12cm} |}
    \hline
	Requirement & Bringing up the program menu \\ \hline
	Conditions & The Patient, who has completed calibration, is wearing the
	Emotiv device. The software is running but in the idle state. \\ \hline
	Procedure &
\begin{enumerate}
	\item The Patient thinks the strong thought that has been mapped to menu
	    activation.
\end{enumerate} \\ \hline
	Expected Results & The menu is displayed on the computer screen. \\ \hline
	Actual Results & \vspace{1cm} \\ \hline
\end{tabular}

\hfill \\

\subsection{SR215 - Perform Home Automation Action}
\subsubsection{T215-01 - Turn Lights On}
\begin{tabular}{| l | p{12cm} |}
    \hline
	Requirement & Turning a connected light on \\ \hline
	Conditions & The software is running and connected to an Insteon light
	control device. The Patient has completed calibration. The menu is
	showing. \\
	\hline
	Procedure &
\begin{enumerate}
	\item The Patient navigates the menu to the desired light device.
	\item The Patient selects the Turn On action.
	\item The Patient confirms the action.
\end{enumerate} \\ \hline
	Expected Results & The light turns on. The program returns to the idle state. \\ \hline
	Actual Results & \vspace{1cm} \\ \hline
\end{tabular}

\subsection{SR215 - Perform Home Automation Action}
\subsubsection{T215-02 - Change Television Channel}
\begin{tabular}{| l | p{12cm} |}
    \hline
	Requirement & Changing the channel on a connected television \\ \hline
	Conditions & The software is running and connected to an Insteon
	television control device; the television is on. The Patient has completed
	calibration. The menu is showing. \\
	\hline
	Procedure &
\begin{enumerate}
	\item The Patient navigates the menu to the desired television device.
	\item The Patient selects the Change Channel action.
	\item The Patient selects a channel number.
	\item The patient confirms the action.
\end{enumerate} \\ \hline
	Expected Results & The television changes to the selected channel. The
    program returns to the idle state. \\ \hline
	Actual Results & \vspace{1cm} \\ \hline
\end{tabular}

\hfill \\

\newpage

\section{Usability Testing}
%Describe the usability test you plan to perform at each phase of you system development to access the usability of the system, including both tests with and without users involved.]
\subsection{Tasks}

Several of the tests in this section refer to a series of tasks to be
completed by a tester. Unless otherwise specified, this series of tasks shall
include the following:

\begin{enumerate}
	\item Calibrate for user
	\item Turn lights on and off
	\item Turn television on, change channel, change volume, turn off.
\end{enumerate}

\subsection{Development/Prototyping Phase}
\begin{tabular}{| l | p{12cm}  |}
    \hline
	Phases & Prototyping/Development, Testing, Final \\ \hline
	Requirement & Software Functionality \\ \hline
	Test Procedure & A number of testers are selected and briefed on the
	system. This briefing includes intended use, target users, and basic
	system functionality. The tester is told that they may interact with the
	system by thinking their accept thought, and that doing so will perform
	the action that is currently indicated on the screen. The tester then
	completes a series of tasks using the system while speaking their thought
	process out loud. The team member monitoring the test takes notes and asks
	the user for more feedback when necessary. It is important that the team
	member does not interfere or help the tester in completing their tasks.
	The goal of this test is to ensure that the system is intuitive for the
	user to use on a basic level and that the system behaves in an expected
	manner.
	\\ \hline
\end{tabular}

\subsection{Testing Phase}
\begin{tabular}{| l | p{12cm}  |}
    \hline
	Phases & Testing, Final \\ \hline
	Requirement & Aesthetic Usability \\ \hline
	Test Procedure & A tester is briefed on the system, as above. The tester
	is then told to turn a light on. The tester is then told to set the light
	level to 50 percent. The team member monitoring the test asks the tester
	for feedback in regards to the system's aesthetics and usability. This
	test ensures that the components of the system are readable and are clear
	and easy to understand. This test should be done in the latter stages of
	the testing phase or in the final phase.
	\\ \hline
\end{tabular}

\begin{tabular}{| l | p{12cm}  |}
    \hline
	Phases & Testing, Final \\ \hline
	Requirement & Error Prevention and Recovery \\ \hline
	Test Procedure & The tester is briefed on the system, as above, and given
	a set of tasks to complete that intentionally cause errors. Once the error
	occurs, the team member will look for feedback from the tester on how they
	would intuitively recover from the error shown. After the error-inducing
	task is complete, the team member asks the tester what they think caused
	the error to be shown and what could they do in the future to prevent it.
	This test can be completed during any phase, but should be more heavily
	focused on during the testing phase.
	\\ \hline
\end{tabular}

\subsection{Final Phase}
\begin{tabular}{| l | p{12cm}  |}
    \hline
	Phases & Prototyping/Development, Testing, Final \\ \hline
	Requirement & General Usability \\ \hline
	Test Procedure & A tester is briefed on the system, as above. A team member
	monitoring the test asks the tester to complete a series of tasks using
	the system. As the tester moves through each step to complete the task,
	they are asked to vocally expresses their thought process and express
	anything that they think is unclear or have an issue with. This can
	include confusing system behavior, unclear user interface design,
	counterintuitive design, missing functionality, ease of use, or anything
	else the user feels would improve the usability of the system. This test
	can be completed at any phase of the development process in order to catch
	any usability issues early. However, there should be a large focus on this
	test towards the final phase when the system is nearing release. \\ \hline
\end{tabular}

\begin{tabular}{| l | p{12cm}  |}
    \hline
	Phases & Final \\ \hline
	Requirement & Consistency and Standards \\ \hline
	Test Procedure & A tester is briefed on the system, as above. A team member
	monitoring the test asks the tester to complete a series of tasks using
	the system. As the tester moves through each step to complete the task,
	the team member looks for feedback regarding the consistency of the
	system. This includes design consistency as well as consistency in
	functionality. This test also covers general system standards. This means
	ensuring the layout, functionality, and design follow the standards set
	for similar user applications. This test can be completed earlier, but
	should be included in the final phase.  \\ \hline
\end{tabular}

\begin{tabular}{| l | p{12cm}  |}
    \hline
	Phases & Final \\ \hline
	Requirement & Documentation \\ \hline
	Test Procedure & A tester is not briefed on the system and asked to read
	and review the documentation on the system. The team member distributing
	the test looks for feedback regarding the documentation itself. This
	includes factors such as clarity, completeness, and overall usefulness.
	Then, once the tester feels they are comfortable enough with the
	documentation, they are asked to complete a series of tasks. This test
	ensures that the documentation is clear, concise, and helpful to the user.
	This test should be done in the final phase near the release when all the
	documentation is complete. \\ \hline

\end{tabular}


\newpage

\section{Performance Testing}
%[In the SRS, if there is any specified performance requirements in Section 3.3, you need to describe the performance requirement and how you plan to test if such requirement is satisfied.]
%Can use multiples subsections if need be
\subsection{SR305}
\begin{tabular}{| l | p{12cm}  |}
    \hline
	Requirement & The system must use less than 512 MB of RAM at all times while running. \\ \hline
	Test Procedure & Start the system on a machine meeting the minimum system
	requirements. Monitor the memory usage of the program during the
	following:
	\begin{enumerate}
	    \item User calibration
	    \item Menu selection and automated home response
	    \item Idle time
	\end{enumerate}
	Verify that memory usage remains below 512 MB. \\ \hline
\end{tabular}
\subsection{SR310}
\begin{tabular}{| l | p{12cm}  |}
    \hline
	Requirement & The system must use less than 5 GB of persistent storage
	space when installed. \\ \hline
	Test Procedure & Install the system on a machine meeting the minimum
	system requirements and check the size of the installation directory. Make
	sure it is below 5 GB. \\ \hline
\end{tabular}
\subsection{SR315}
\begin{tabular}{| l | p{12cm}  |}
    \hline
	Requirement & The system must run smoothly on a 3.2 GHz Intel i3 processor. \\ \hline
	Test Procedure & Install the system on a machine with these specifications
	and meeting the other minimum system requirements. Perform the following
	tasks and verify that it runs smoothly (i.e. without stuttering or
	becoming unresponsive):
	\begin{enumerate}
	    \item User calibration
	    \item Menu selection and automated home response
	    \item Idle time
	\end{enumerate}
	\\ \hline
\end{tabular}
\subsection{SR320}
%CHECK THIS.
%Not sure if the following 2 are necessary
% FIXME Should probably have something down, even if it's trivial like this.
% Is this OK?
\begin{tabular}{| l | p{12cm}  |}
    \hline
	Requirement & The system will support one and only one terminal. \\ \hline
	Test Procedure & Launch the program. Verify that it runs on one machine. \\ \hline
\end{tabular}
\subsection{SR325}
\begin{tabular}{| l | p{12cm}  |}
    \hline
	Requirement & The system will support one and only one user at a time. \\ \hline
	Test Procedure & Launch the program. Verify that it runs for one user. \\ \hline
\end{tabular}
\subsection{SR330}
\begin{tabular}{| l | p{12cm}  |}
    \hline
    % XXX Come to think of it, this one is hard to verify since it depends on
    % a "trained user."
    % XXX What kind of menu selections?
	Requirement & 90\% of time, menu selection must take less than 5 seconds. \\ \hline
	Test Procedure & Time a trained user making a sequence of at least 20 menu selections. Verify that at least 90\% of the selections are made correctly in less than 5 seconds. \\ \hline
\end{tabular}
\subsection{SR335}
\begin{tabular}{| l | p{12cm}  |}
    \hline
	Requirement & A trained user must be able to select 10 options within 2 minutes. \\ \hline
	Test Procedure & Time a trained user making a sequence of 10 menu selections. Verify that this takes under 2 minutes. \\ \hline
\end{tabular}

\newpage

\section{Compliance Testing}
%[If the system must comply with certain functional or technical standards, describe the aspects of the standard that must be complied here and how you plan to test such compliance.]
% NOTE: We dont have standards with insteon i think. Just write stuff about testing the API we wrap theirs with?

There are no functional or technical standards with which the system must
comply.

\newpage
\section*{\centering Table of Contributions}
\begin{tabular}{| l | l | l | l |}
    \hline
     Section & Writing & Editing \\
    \hline \hline
		1 & Joe Muoio & Michael Conway \\ \hline
		2 & Kevin Zakszewski & Michael Conway \\ \hline
		3 & Sam Bever & Michael Conway \\ \hline
		4 & Joe Muoio & Michael Conway \\ \hline
		5 & Michael Conway & Michael Conway \\ \hline
\end{tabular}
\newpage
\noindent I certify that:
\begin{itemize}
\item This paper/project/exam is entirely my own work.
\item I have not quoted the words of any other person from a printed source or a website without indicating what has been quoted and providing an appropriate citation.
\item I have not submitted this paper / project to satisfy the requirements of any other course.
\end{itemize}

\vspace{1cm}
\noindent\makebox[\textwidth][l]{
Signature:
\makebox[5cm][l] {\underline{Samuel Bever}} 
\ \ Date:
\makebox[4cm][l] {\underline{\required\today}} 
}


\vspace{0.5cm}
\noindent\makebox[\textwidth][l]{
Signature:
\makebox[5cm][l] {\underline{Michael Conway}} 
\ \ Date:
\makebox[4cm][l] {\underline{\required\today}} 
}

\vspace{0.5cm}
\noindent\makebox[\textwidth][l]{
Signature:
\makebox[5cm][l] {\underline{Joe Muoio}} 
\ \ Date:
\makebox[4cm][l] {\underline{\required\today}} 
}

\vspace{0.5cm}
\noindent\makebox[\textwidth][l]{
Signature:
\makebox[5cm][l] {\underline{Kyle Patron}} 
\ \ Date:
\makebox[4cm][l] {\underline{\required\today}} 
}

\vspace{0.5cm}
\noindent\makebox[\textwidth][l]{
Signature:
\makebox[5cm][l] {\underline{Kevin Zakszewski}} 
\ \ Date:
\makebox[4cm][l] {\underline{\required\today}} 
}

\vspace{\fill}
\subsection*{Grading}
The grade is given on the basis of quality, clarity, presentation, completeness, and writing of each section in the report. This is the grade of the group. Individual grades will be assigned at the end of the term when peer reviews are collected.
\end{document}
