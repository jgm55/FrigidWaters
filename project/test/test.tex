\documentclass{article}

\usepackage[margin=1in]{geometry}
\usepackage{parskip}
\usepackage{datetime}
\usepackage{url}
\usepackage{hyperref}
\usepackage[utf8]{inputenc}
\usepackage{graphicx}

\newdateformat{required}{\twodigit{\THEMONTH}/\twodigit{\THEDAY}/\THEYEAR}
\raggedbottom
\begin{document}

\begin{titlepage}
    \begin{center}
        \begin{huge}
        House in Your Head \\[1cm]
        Team G1-FrigidWaters \\[2.2cm]
        { \bfseries Test Specification } \\[1cm]
        Cycle \# 2\\[2.2cm]
        Date: \required\today\\[1cm]
        \end{huge}
    \end{center}
    \null \vfill
    \begin{large}
        Team Members: \\[0.5cm]
        Name: Samuel Bever\\[0.5cm]
        Name: Michael Conway\\[0.5cm]
        Name: Joseph Muoio\\[0.5cm]
        Name: Kyle Patron\\[0.5cm]
        Name: Kevin Zakszewski
    \end{large}
\end{titlepage}
\section*{\centering Table of Contents}
\makeatletter
\@starttoc{toc}
\newcommand{\hsubsubsection}{
\@startsection{subsubsection}{3}{\z@}%
                                     {-3.25ex\@plus -1ex \@minus -.2ex}%
                                     {-1.5ex \@plus -.2ex}% Formerly 1.5ex \@plus .2ex
                                     {R\normalfont\normalsize}}
\newcommand{\hparagraph}{
\@startsection{paragraph}{4}{\z@}%
                                     {-3.25ex\@plus -1ex \@minus -.2ex}%
                                     {-1.5ex \@plus -.2ex}% Formerly 1.5ex \@plus .2ex
                                     {R\normalfont\normalsize}}
\newcommand{\hsubparagraph}{
\@startsection{subparagraph}{5}{\z@}%
                                     {-3.25ex\@plus -1ex \@minus -.2ex}%
                                     {-1.5ex \@plus -.2ex}% Formerly 1.5ex \@plus .2ex
                                     {R\normalfont\normalsize}}
\setcounter{secnumdepth}{5}
\makeatother
\newpage
 

\section{Introduction}

\subsection{Purpose}

The purpose of this document is to document the test specifications for the House in Your Head Project. This is written for test engineers so that they know which tests to perform to make sure the system is functioning properly. This section is the introduction. The following sections lay out Functional Tests, Usability Tests, Performance Tests, and Compliance Tests.

\subsection{References}

% TODO: Finish this when rest is written
% [If this report is citing information from other documents such as the prior reports, list the references here. For example, the test specification report may make reference to the SRS or the SDS reports.]


\newpage

\section{Functional Test Specifications}
% Every Subsection follows the below example

% Template
%\subsection{Requirement id - Requirement Name}
%\subsubsection{TestID - Test Name}
%\begin{tabular}{| l | l |}
%    \hline
%	Objective & [Define what is going to be tested.  Use the same terminology as you define the functions.] \\ \hline
%	Conditions and Procedures & [Define the conditions that are needed for the test.  E.g. state of the system, user actions, content of data files, values of data elements, input values etc.] \\ \hline
%	Expected Results & [Define the expected result when the test is executed.  E.g. output values, system responses, event triggering, etc.] \\ \hline
%	Actual Results & [This field provides a place for the test engineers to record the results of the test.] \\ \hline
%\end{tabular}

\begin{tabular}{| l | p{12cm} |}
    \hline
	Objective & [Define what is going to be tested.  Use the same terminology as you define the functions.] \\ \hline
	Conditions and Procedures & [Define the conditions that are needed for the test.  E.g. state of the system, user actions, content of data files, values of data elements, input values etc.] \\ \hline
	Expected Results & [Define the expected result when the test is executed.  E.g. output values, system responses, event triggering, etc.] \\ \hline
	Actual Results & [This field provides a place for the test engineers to record the results of the test.] \\ \hline
\end{tabular}

\hfill \\

\newpage

\section{Usability Testing}
%Describe the usability test you plan to perform at each phase of you system development to access the usability of the system, including both tests with and without users involved.]
\subsection{Development/Prototyping Phase}
\begin{tabular}{| l | p{12cm}  |}


    \hline
	Phases & Prototyping/Development, Testing, Final \\ \hline
	Requirement & Software Functionality \\ \hline
	Test Procedure & A number of testers are selected and briefed on the system. This briefing includes intended use, target users, and basic system functionality. The tester then completes a series of tasks using out system while speaking their thought process out loud. The team member monitoring the test takes notes and asks the user for more feedback when necessary. It is important that the team member does not interfere or help the tester in completing their tasks. The goal of this test is to ensure that the system is intuitive for the user to use on a basic level and that the system behaves in an expected manner.   \\ \hline

\end{tabular}

\subsection{Testing Phase}
\begin{tabular}{| l | p{12cm}  |}
    \hline
	Phases & Testing, Final \\ \hline
	Requirement & Aesthetic Usability \\ \hline
	Test Procedure & A tester is briefed on the system. This briefing includes intended use, target users, and basic system functionality. The tester than completes a series of simple tasks throughout the entire system. The team member monitoring the test asks the tester for feedback in regards to the system's aesthetics. This test ensures that the components of the system are readable and are clear and easy to understand. This test should be done in the latter stages of the testing phase or in the final phase. \\ \hline

	Phases & Testing, Final \\ \hline
	Requirement & Error Prevention and Recovery \\ \hline
	Test Procedure & The tester is given a set of tasks to complete that intentionally cause errors. Once the error occurs, the team member will look for feedback from the tester on how they would intuitively recover from the error shown. After the error-inducing task is complete, the team member asks the tester what they think caused the error to be shown and what could they do in the future to prevent it. This test can be completed during any phase, but should be more heavily focused on during the testing phase. \\ \hline


\end{tabular}

\subsection{Final Phase}
\begin{tabular}{| l | p{12cm}  |}
    \hline
	Phases & Prototyping/Development, Testing, Final \\ \hline
	Requirement & General Usability \\ \hline
	Test Procedure & A tester is briefed on the system. This briefing includes intended use, target users, and basic system functionality. A team member monitoring the test asks the tester to complete a series of tasks using our system. As the tester moves through each step to complete the task, they are asked to vocally expresses their thought process and express anything that they think is unclear or have an issue with. This can include confusing system behavior, unclear user interface design, counterintuitive design, missing functionality, ease of use, or anything else the user feels would improve the usability of the system. This test can be completed at any phase of the development process in order to catch any usability issues early. However, there should be a large focus on this test towards the final phase when the system is nearing release. \\ \hline

	Phases & Final \\ \hline
	Requirement & Consistency and Standards \\ \hline
	Test Procedure & A tester is briefed on the system. This briefing includes intended use, target users, and basic system functionality. A team member monitoring the test asks the tester to complete a series of tasks using our system. As the tester moves through each step to complete the task, the team member looks for feedback regarding the consistency of the system. This includes design consistency as well as consistency in functionality. This test also covers general system standards. This means ensuring the layout, functionality, and design follow the standards set for similar user applications. This test can be completed earlier, but should be included in the final phase.  \\ \hline

	Phases & Final \\ \hline
	Requirement & Documentation \\ \hline
	Test Procedure & A tester is not briefed on the system and asked to read and review the documentation on the system. The team member distributing the test looks for feedback regarding the documentation itself. This includes factors such as clarity, completeness, and overall usefulness. Then, once the test feels they are comfortable enough with the documentation, they are asked to complete a series of tasks. This test ensures that our documentation is clear, concise, and helpful to the user. This test should be done in the final phase near the release when all the documentation is complete. \\ \hline

\end{tabular}


\newpage

\section{Performance Testing}
%[In the SRS, if there is any specified performance requirements in Section 3.3, you need to describe the performance requirement and how you plan to test if such requirement is satisfied.]
%Can use multiples subsections if need be
\subsection{SR305}
\begin{tabular}{| l | p{12cm}  |}
    \hline
	Requirement & The system must use less than 512 MB of RAM at all times while running. \\ \hline
	Test Procedure & Open up Task Manager and check the Memory usage of the program. \\ \hline
\end{tabular}
\subsection{SR310}
\begin{tabular}{| l | p{12cm}  |}
    \hline
	Requirement & The system must use less than 5 GB of persistent storage space when installed. \\ \hline
	Test Procedure & Check the size of the folder that holds all the software for the House in Your Head project. Make sure it is below 5 GB. \\ \hline
\end{tabular}
\subsection{SR315}
\begin{tabular}{| l | p{12cm}  |}
    \hline
	Requirement & The system must run smoothly on a 3.2 GHz Intel i3 processor. \\ \hline
	Test Procedure & Launch the House in Your Head Software and make sure there is no stuttering or lag. \\ \hline
\end{tabular}
\subsection{SR320}
%CHECK THIS.
%Not sure if the following 2 are necessary
\begin{tabular}{| l | p{12cm}  |}
    \hline
	Requirement & The system will support one and only one terminal. \\ \hline
	Test Procedure & Launch the program. \\ \hline
\end{tabular}
\subsection{SR325}
\begin{tabular}{| l | p{12cm}  |}
    \hline
	Requirement & The system will support one and only one user at a time. \\ \hline
	Test Procedure & Launch the program. \\ \hline
\end{tabular}
\subsection{SR330}
\begin{tabular}{| l | p{12cm}  |}
    \hline
	Requirement & 90\% of time, menu selection must take less than 5 seconds. \\ \hline
	Test Procedure & Time a user selecting options. Record all the selections for a few standard use cases. Use statistics to make sure 90\% of the selections happen under the 5 second threshold. \\ \hline
\end{tabular}
\subsection{SR335}
\begin{tabular}{| l | p{12cm}  |}
    \hline
	Requirement & A trained user must be able to select 10 options within 2 minutes. \\ \hline
	Test Procedure & Record a user who has used the device and software going through a use case where the user selects ten options to complete the task. Verify this takes under 2 minutes. \\ \hline
\end{tabular}

\newpage

\section{Compliance Testing}
%[If the system must comply with certain functional or technical standards, describe the aspects of the standard that must be complied here and how you plan to test such compliance.]
% NOTE: We dont have standards with insteon i think. Just write stuff about testing the API we wrap theirs with?
\subsection{Properly Integrate with Insteon}
\begin{tabular}{| l | p{12cm} |}
    \hline
	Requirement & stuff \\ \hline
	Test Procedure & stuff \\ \hline
\end{tabular}

\newpage
\section*{\centering Table of Contributions}
\begin{tabular}{| l | l | l | l |}
    \hline
     & Section & Writing & Editing \\
    \hline \hline
		1 & 2.2, 2.6, 3.3, 3.4, 3.6 & Kyle Patron & Michael Conway\\ \hline
		2 & 1, 2.1, 3.1 & Joe Muoio & Michael Conway\\ \hline
		3 & 2.4, 2.5 & Sam Bever & Michael Conway\\ \hline
		4 & 3.2 & Michael Conway & Kevin Zakszewski \\ \hline
		5 & 1, 2.3 & Kevin Zakszewski & Michael Conway \\ \hline
\end{tabular}
\newpage
\noindent I certify that:
\begin{itemize}
\item This paper/project/exam is entirely my own work.
\item I have not quoted the words of any other person from a printed source or a website without indicating what has been quoted and providing an appropriate citation.
\item I have not submitted this paper / project to satisfy the requirements of any other course.
\end{itemize}

\vspace{1cm}
\noindent\makebox[\textwidth][l]{
Signature:
\makebox[5cm][l] {\underline{Samuel Bever}} 
\ \ Date:
\makebox[4cm][l] {\underline{\required\today}} 
}


\vspace{0.5cm}
\noindent\makebox[\textwidth][l]{
Signature:
\makebox[5cm][l] {\underline{Michael Conway}} 
\ \ Date:
\makebox[4cm][l] {\underline{\required\today}} 
}

\vspace{0.5cm}
\noindent\makebox[\textwidth][l]{
Signature:
\makebox[5cm][l] {\underline{Joe Muoio}} 
\ \ Date:
\makebox[4cm][l] {\underline{\required\today}} 
}

\vspace{0.5cm}
\noindent\makebox[\textwidth][l]{
Signature:
\makebox[5cm][l] {\underline{Kyle Patron}} 
\ \ Date:
\makebox[4cm][l] {\underline{\required\today}} 
}

\vspace{0.5cm}
\noindent\makebox[\textwidth][l]{
Signature:
\makebox[5cm][l] {\underline{Kevin Zakszewski}} 
\ \ Date:
\makebox[4cm][l] {\underline{\required\today}} 
}

\vspace{\fill}
\subsection*{Grading}
The grade is given on the basis of quality, clarity, presentation, completeness, and writing of each section in the report. This is the grade of the group. Individual grades will be assigned at the end of the term when peer reviews are collected.
\end{document}
