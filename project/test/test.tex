\documentclass{article}

\usepackage[margin=1in]{geometry}
\usepackage{parskip}
\usepackage{datetime}
\usepackage{url}
\usepackage{hyperref}
\usepackage[utf8]{inputenc}
\usepackage{graphicx}

\newdateformat{required}{\twodigit{\THEMONTH}/\twodigit{\THEDAY}/\THEYEAR}
\raggedbottom
\begin{document}

\begin{titlepage}
    \begin{center}
        \begin{huge}
        House in Your Head \\[1cm]
        Team G1-FrigidWaters \\[2.2cm]
        { \bfseries Test Specification } \\[1cm]
        Cycle \# 2\\[2.2cm]
        Date: \required\today\\[1cm]
        \end{huge}
    \end{center}
    \null \vfill
    \begin{large}
        Team Members: \\[0.5cm]
        Name: Samuel Bever\\[0.5cm]
        Name: Michael Conway\\[0.5cm]
        Name: Joseph Muoio\\[0.5cm]
        Name: Kyle Patron\\[0.5cm]
        Name: Kevin Zakszewski
    \end{large}
\end{titlepage}
\section*{\centering Table of Contents}
\makeatletter
\@starttoc{toc}
\newcommand{\hsubsubsection}{
\@startsection{subsubsection}{3}{\z@}%
                                     {-3.25ex\@plus -1ex \@minus -.2ex}%
                                     {-1.5ex \@plus -.2ex}% Formerly 1.5ex \@plus .2ex
                                     {R\normalfont\normalsize}}
\newcommand{\hparagraph}{
\@startsection{paragraph}{4}{\z@}%
                                     {-3.25ex\@plus -1ex \@minus -.2ex}%
                                     {-1.5ex \@plus -.2ex}% Formerly 1.5ex \@plus .2ex
                                     {R\normalfont\normalsize}}
\newcommand{\hsubparagraph}{
\@startsection{subparagraph}{5}{\z@}%
                                     {-3.25ex\@plus -1ex \@minus -.2ex}%
                                     {-1.5ex \@plus -.2ex}% Formerly 1.5ex \@plus .2ex
                                     {R\normalfont\normalsize}}
\setcounter{secnumdepth}{5}
\makeatother
\newpage
 

\section{Introduction}

\subsection{Purpose}

The purpose of this document is to document the test specifications for the House in Your Head Project. This is written for test engineers so that they know which tests to perform to make sure the system is functioning properly. This section is the introduction. The following sections lay out Functional Tests, Usability Tests, Performance Tests, and Compliance Tests.

\subsection{References}

% TODO: Finish this when rest is written
% [If this report is citing information from other documents such as the prior reports, list the references here. For example, the test specification report may make reference to the SRS or the SDS reports.]


\newpage

\section{Functional Test Specifications}
% Every Subsection follows the below example

% Template
%\subsection{Requirement id - Requirement Name}
%\subsubsection{TestID - Test Name}
%\begin{tabular}{| l | l |}
%    \hline
%	Objective & [Define what is going to be tested.  Use the same terminology as you define the functions.] \\ \hline
%	Conditions and Procedures & [Define the conditions that are needed for the test.  E.g. state of the system, user actions, content of data files, values of data elements, input values etc.] \\ \hline
%	Expected Results & [Define the expected result when the test is executed.  E.g. output values, system responses, event triggering, etc.] \\ \hline
%	Actual Results & [This field provides a place for the test engineers to record the results of the test.] \\ \hline
%\end{tabular}

\begin{tabular}{| l | p{12cm} |}
    \hline
	Objective & [Define what is going to be tested.  Use the same terminology as you define the functions.] \\ \hline
	Conditions and Procedures & [Define the conditions that are needed for the test.  E.g. state of the system, user actions, content of data files, values of data elements, input values etc.] \\ \hline
	Expected Results & [Define the expected result when the test is executed.  E.g. output values, system responses, event triggering, etc.] \\ \hline
	Actual Results & [This field provides a place for the test engineers to record the results of the test.] \\ \hline
\end{tabular}

\hfill \\

\newpage

\section{Usability Testing}
%Describe the usability test you plan to perform at each phase of you system development to access the usability of the system, including both tests with and without users involved.]
\subsection{Project Phase}
\begin{tabular}{| l | p{12cm}  |}
    \hline
	Requirement & stuff \\ \hline
	Test Procedure & stuff \\ \hline
\end{tabular}

\newpage

\section{Performance Testing}
%[In the SRS, if there is any specified performance requirements in Section 3.3, you need to describe the performance requirement and how you plan to test if such requirement is satisfied.]
%Can use multiples subsections if need be
\subsection{SR305}
\begin{tabular}{| l | p{12cm}  |}
    \hline
	Requirement & The system must use less than 512 MB of RAM at all times while running. \\ \hline
	Test Procedure & Start the system on a machine meeting the minimum system
	requirements. Monitor the memory usage of the program during the
	following:
	\begin{enumerate}
	    \item User calibration
	    \item Menu selection and automated home response
	    \item Idle time
	\end{enumerate}
	Verify that memory usage remains below 512 MB. \\ \hline
\end{tabular}
\subsection{SR310}
\begin{tabular}{| l | p{12cm}  |}
    \hline
	Requirement & The system must use less than 5 GB of persistent storage
	space when installed. \\ \hline
	Test Procedure & Install the system on a machine meeting the minimum
	system requirements and check the size of the installation directory. Make
	sure it is below 5 GB. \\ \hline
\end{tabular}
\subsection{SR315}
\begin{tabular}{| l | p{12cm}  |}
    \hline
	Requirement & The system must run smoothly on a 3.2 GHz Intel i3 processor. \\ \hline
	Test Procedure & Install the system on a machine with these specifications
	and meeting the other minimum system requirements. Perform the following
	tasks and verify that it runs smoothly (i.e. without stuttering or
	becoming unresponsive):
	\begin{enumerate}
	    \item User calibration
	    \item Menu selection and automated home response
	    \item Idle time
	\end{enumerate}
	\\ \hline
\end{tabular}
\subsection{SR320}
%CHECK THIS.
%Not sure if the following 2 are necessary
% FIXME Should probably have something down, even if it's trivial like this.
% Is this OK?
\begin{tabular}{| l | p{12cm}  |}
    \hline
	Requirement & The system will support one and only one terminal. \\ \hline
	Test Procedure & Launch the program. Verify that it runs on one machine. \\ \hline
\end{tabular}
\subsection{SR325}
\begin{tabular}{| l | p{12cm}  |}
    \hline
	Requirement & The system will support one and only one user at a time. \\ \hline
	Test Procedure & Launch the program. Verify that it runs for one user. \\ \hline
\end{tabular}
\subsection{SR330}
\begin{tabular}{| l | p{12cm}  |}
    \hline
    % XXX Come to think of it, this one is hard to verify since it depends on
    % a "trained user."
    % XXX What kind of menu selections?
	Requirement & 90\% of time, menu selection must take less than 5 seconds. \\ \hline
	Test Procedure & Time a trained user making a sequence of at least 20 menu selections. Verify that at least 90\% of the selections are made correctly in less than 5 seconds. \\ \hline
\end{tabular}
\subsection{SR335}
\begin{tabular}{| l | p{12cm}  |}
    \hline
	Requirement & A trained user must be able to select 10 options within 2 minutes. \\ \hline
	Test Procedure & Time a trained user making a sequence of 10 menu selections. Verify that this takes under 2 minutes. \\ \hline
\end{tabular}

\newpage

\section{Compliance Testing}
%[If the system must comply with certain functional or technical standards, describe the aspects of the standard that must be complied here and how you plan to test such compliance.]
% NOTE: We dont have standards with insteon i think. Just write stuff about testing the API we wrap theirs with?

There are no functional or technical standards with which the system must
comply.

\newpage
\section*{\centering Table of Contributions}
\begin{tabular}{| l | l | l | l |}
    \hline
     Section & Writing & Editing \\
    \hline \hline
		1 & Joe Muoio & Michael Conway \\ \hline
		2 & Kevin Zakszewski & Michael Conway \\ \hline
		3 & Sam Bever & Michael Conway \\ \hline
		4 & Joe Muoio & Michael Conway \\ \hline
		5 & Michael Conway & Michael Conway \\ \hline
\end{tabular}
\newpage
\noindent I certify that:
\begin{itemize}
\item This paper/project/exam is entirely my own work.
\item I have not quoted the words of any other person from a printed source or a website without indicating what has been quoted and providing an appropriate citation.
\item I have not submitted this paper / project to satisfy the requirements of any other course.
\end{itemize}

\vspace{1cm}
\noindent\makebox[\textwidth][l]{
Signature:
\makebox[5cm][l] {\underline{Samuel Bever}} 
\ \ Date:
\makebox[4cm][l] {\underline{\required\today}} 
}


\vspace{0.5cm}
\noindent\makebox[\textwidth][l]{
Signature:
\makebox[5cm][l] {\underline{Michael Conway}} 
\ \ Date:
\makebox[4cm][l] {\underline{\required\today}} 
}

\vspace{0.5cm}
\noindent\makebox[\textwidth][l]{
Signature:
\makebox[5cm][l] {\underline{Joe Muoio}} 
\ \ Date:
\makebox[4cm][l] {\underline{\required\today}} 
}

\vspace{0.5cm}
\noindent\makebox[\textwidth][l]{
Signature:
\makebox[5cm][l] {\underline{Kyle Patron}} 
\ \ Date:
\makebox[4cm][l] {\underline{\required\today}} 
}

\vspace{0.5cm}
\noindent\makebox[\textwidth][l]{
Signature:
\makebox[5cm][l] {\underline{Kevin Zakszewski}} 
\ \ Date:
\makebox[4cm][l] {\underline{\required\today}} 
}

\vspace{\fill}
\subsection*{Grading}
The grade is given on the basis of quality, clarity, presentation, completeness, and writing of each section in the report. This is the grade of the group. Individual grades will be assigned at the end of the term when peer reviews are collected.
\end{document}
