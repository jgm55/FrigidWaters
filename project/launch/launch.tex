\documentclass{report}

\usepackage[margin=1in]{geometry}
\usepackage{parskip}
\usepackage{datetime}
\usepackage{pgfgantt}

\newdateformat{required}{\twodigit{\THEMONTH}/\twodigit{\THEDAY}/\THEYEAR}

\begin{document}

\begin{titlepage}
    \begin{center}
        \begin{huge}
        ALS Project \\[1cm]
        Team Doocy \\[2.2cm]
        { \bfseries Launch Report } \\[1cm]
        Cycle \# 1\\[2.2cm]
        Date: \required\today\\[1cm]
        \end{huge}
    \end{center}
    \null \vfill
    \begin{large}
        Team Members: \\[0.5cm]
        Name: Samuel Bever\\[0.5cm]
        Name: Michael Conway\\[0.5cm]
        Name: Joseph Muoio\\[0.5cm]
        Name: Kyle Patron\\[0.5cm]
        Name: Kevin Zakszewski
    \end{large}
\end{titlepage}

\section*{\centering The System/Product}

\subsection*{System/Product Name}

% TODO

\subsection*{Introduction}

Amyotrophic lateral sclerosis (ALS) (also called ``Lou Gehrig's Disease'') is
a neurodegenerative disease affecting the brain and spinal cord. The motor
neurons progressively degenerate and die off, leading to the loss of muscle
control and death. Affected individuals in later stages of the disease may
become paralyzed, losing the ability to interact with the world in even simple
ways like turning on a light switch. Fully paralyzed patients are unable to
move or even to speak, making communication impossible, let alone interfacing
with conventional assistive technologies.

The goal of the project is to build a system which people afflicted with ALS
can use to interact with their houses through Home Automation Systems (HAS).
This includes controlling the thermostat, lights, stereo, and television as
well as interfacing with social media. By having a Brain Computer Interface
(BCI) to their automated house, individuals with ALS can still interact with
their environment despite full paralysis. This will build off of a project
from last year. The interface will be improved and new devices will be
experimented with.

% At least one thing that sounds rather similar (Thought-Wired)
% http://www.thought-wired.com/

\subsection*{Highlighted Features}

\begin{itemize}
\item Allow users with ALS to perform basic household tasks without needing to
    move
\item Integrate different BCI devices with different HAS
\end{itemize}

\subsection*{Sponsor or Proxy User}

Our outside stakeholder is \textbf{Sara Feldman}, PT, DPT, ATP, MDA/ALS Center
of Hope, Drexel University College of Medicine. We will be meeting with Dr.
Feldman to establish the specific goals of the project and the requirements
that it must meet to be usable for ALS patients.

\subsection*{Issues}

There are many potential features of this product that would be useful to a
patient suffering from ALS. 

% BCIs not yet precise
%BCIs are probably not the most precise devices. They also need to be trained
%for a particular user. Making this process as easy as possible and as
%accurate as possible will be a big challenge.
% We would need a ref for this

% Different ones out there (confirm? at least different levels of
% invasiveness); design to handle all of them with appropriate module?
Another challenge will be integrating different BCI devices with different
types of home control systems. There are multiple different BCI devices and
multiple different home automation systems. Our goal is for our software
design to be flexible enough to handle the cases of different BCI devices
integrating with different home control systems.

However, the different assumptions made by these different products may limit
our ability to do so effectively and still ship a useful product. For example,
HASs targeting the general population may not cover some simple tasks that are
nonetheless impossible for a paralyzed patient.

\newpage
\section*{\centering The Team}

\subsection*{Team Name}

% TODO

\subsection*{Team Members and their specialties}

Samuel Bever

Michael Conway has experience and interests in the areas of algorithms,
programming languages and computer security. % TODO project role

Joseph Muoio

Kyle Patron

Kevin Zakszewski

% Roles:

% Team leader
% ...

\subsection*{Team Communication}

Our team will be meeting at least once a week to discuss our progress, outline
our tasks for that week, and address any other issues. If necessary, more
meetings will be scheduled. Most group communication outside of these meetings
will be done via email. For source code management, we will be using a GitHub
repository, which comes with an issue tracker and a project wiki.

\subsection*{Team Issues}

% Discuss scheduling issues, issues with "specialty" (too many backend, not
% enough front end developers etc), or other forseen issues

\newpage
\section*{\centering The Plan}

\subsection*{Objectives}

Our goals for this first cycle are geared towards gaining the information that
will be vital to the success of the project and preparing a specific plan for
the remaining two cycles. We intend to do the following:
\begin{itemize}
    \item Research the problem space, including the BCI and HAS technologies
        that are available to us
    \item Gather requirements from prospective users
    \item Develop an initial system design
    \item Build a prototype in order to
        \begin{itemize}
            \item test the feasibility of the design
            \item identify the most important challenges to focus on in the
                remaining cycles
        \end{itemize}
    \item Present the prototype to prospective users in order to clearly
        define assumed, unstated or non-obvious requirements
\end{itemize}

\subsection*{Schedule}

\begin{tabular}{| l | l | l |}
    \hline
    Week & Person & Contribution \\
    \hline \hline
    Lorem & ipsum & dolor sit amet \\ \hline
\end{tabular}

\begin{ganttchart}{1}{11}
    \gantttitle{Fall Term}{11} \\
    \gantttitlelist{1,...,11}{1} \\

    % These are all "at the end of week #"
    \ganttbar{Write launch report}{2}{2} \\
    \ganttlinkedmilestone{Launch Report 1}{2} \\
    \ganttbar{Gather requirements}{3}{5} \\
    \ganttlinkedmilestone{SRS 1}{5} \\
    \ganttbar{Design}{5}{7} \\
    \ganttlinkedmilestone{SDS 1}{7} \\
    \ganttbar{Implement prototype}{7}{9} \\
    \ganttlinkedmilestone{Prototype 1}{9} \\
    \ganttbar{Prepare presentation}{8}{10} \\
    \ganttlinkedmilestone{Presentation 1}{10}

\end{ganttchart}

\newpage
\section*{\centering Table of Contributions}
\begin{tabular}{| l | l | l | l |}
    \hline
     & Section & Writing & Editing \\
    \hline \hline
    1 & Project & & \\ \hline
    2 & Team & & \\ \hline
    3 & Plan & & \\ \hline
\end{tabular}
\newpage
\noindent I certify that:
\begin{itemize}
\item This paper/project/exam is entirely my own work.
\item I have not quoted the words of any other person from a printed source or a website without indicating what has been quoted and providing an appropriate citation.
\item I have not submitted this paper / project to satisfy the requirements of any other course.
\end{itemize}

\vspace{1cm}
\noindent\makebox[\textwidth][l]{
Signature:
\makebox[5cm][l] {\hrulefill} 
\ \ Date:
\makebox[4cm][l] {\hrulefill} 
}


\vspace{0.5cm}
\noindent\makebox[\textwidth][l]{
Signature:
\makebox[5cm][l] {\hrulefill} 
\ \ Date:
\makebox[4cm][l] {\hrulefill} 
}

\vspace{0.5cm}
\noindent\makebox[\textwidth][l]{
Signature:
\makebox[5cm][l] {\hrulefill} 
\ \ Date:
\makebox[4cm][l] {\hrulefill} 
}

\vspace{0.5cm}
\noindent\makebox[\textwidth][l]{
Signature:
\makebox[5cm][l] {\hrulefill} 
\ \ Date:
\makebox[4cm][l] {\hrulefill} 
}

\vspace{0.5cm}
\noindent\makebox[\textwidth][l]{
Signature:
\makebox[5cm][l] {\hrulefill} 
\ \ Date:
\makebox[4cm][l] {\hrulefill} 
}
\vspace{\fill}
\subsection*{Grading}
The grade is given on the basis of quality, clarity, presentation, completeness, and writing of each section in the report. This is the grade of the group. Individual grades will be assigned at the end of the term when peer reviews are collected.
\end{document}
