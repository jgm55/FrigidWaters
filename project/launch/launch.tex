\documentclass{report}

\usepackage[margin=1in]{geometry}
\usepackage{parskip}
\usepackage{datetime}
\usepackage{pgfgantt}
\usepackage{url}
\usepackage{hyperref}

\newdateformat{required}{\twodigit{\THEMONTH}/\twodigit{\THEDAY}/\THEYEAR}

\begin{document}

\begin{titlepage}
    \begin{center}
        \begin{huge}
        ALS Project (working title) \\[1cm]
        Team Ten Degree Water \\[2.2cm]
        { \bfseries Launch Report } \\[1cm]
        Cycle \# 1\\[2.2cm]
        Date: \required\today\\[1cm]
        \end{huge}
    \end{center}
    \null \vfill
    \begin{large}
        Team Members: \\[0.5cm]
        Name: Samuel Bever\\[0.5cm]
        Name: Michael Conway\\[0.5cm]
        Name: Joseph Muoio\\[0.5cm]
        Name: Kyle Patron\\[0.5cm]
        Name: Kevin Zakszewski
    \end{large}
\end{titlepage}

\section*{\centering The System/Product}

\subsection*{System/Product Name}

% TODO

\subsection*{Introduction}

Amyotrophic lateral sclerosis (ALS) (also called ``Lou Gehrig's Disease'') is
a neurodegenerative disease affecting the brain and spinal cord. The motor
neurons progressively degenerate and die off, leading to the loss of muscle
control and death. Affected indivuals suffer a spectrum of symptoms
culminating in total lock-in. In this state, the patient is aware but cannot
move their body in any way. They cannot breathe on their own or even blink
their eyes. Because of this, late-stage ALS patients are unable to interface
with conventional assistive technologies.\cite{ALSsource}

The goal of the project is to build a system which people afflicted with ALS
can use to interact with their houses through Home Automation Systems (HAS).
This includes controlling the thermostat, lights, stereo, and television as
well as interfacing with social media. By having a Brain Computer Interface
(BCI) to their automated house, individuals with ALS can still interact with
their environment despite full paralysis. This builds off of a previous
senior design project. The interface will be improved, methods of
input training and processing will be experimented with, and a more robust
array of input options will be used.

% At least one thing that sounds rather similar (Thought-Wired)
% http://www.thought-wired.com/

\subsection*{Highlighted Features}

\begin{itemize}
\item Interpret brain waves
\item Calibrate interface to individual users
\item Allow users with ALS to perform basic household tasks
\end{itemize}

\subsection*{Sponsor or Proxy User}

Our outside stakeholder is \textbf{Sara Feldman}, PT, DPT, ATP, MDA/ALS Center
of Hope, Drexel University College of Medicine. We will be meeting with Dr.
Feldman on an ongoing basis to continually reevaluate the progress and
project goals to ensure that this product will be useful to ALS patients.

\subsection*{Issues}

Brains are complicated, and so expecting them to perform a task in a
deterministic way is not necessarily realistic. This means that we will have
to have a sophisticated signal processing system in place. There is no one
correct way to do this, and since the current method is not robust enough
for general use, we predict that this will occupy a significant portion of
the project development cycle.

Gathering project requirements will present difficulties because we cannot
easily identify with the daily experiences of ALS patients and communicating
directly with late-stage ALS patients is challenging. To circumvent this, we
will have to depend heavily on our contacts at Drexel Medicine, who work
with ALS patients.

\subsection*{Project Costs}
The Emotiv device costs \$300 to \$500. This will be covered by the school
for our development purposes. Before being able to be used, the device will
have to be trained, costing the user time. The software will be fully open
source, so it will be free for users to download and use as long as they
have the Emotiv device.

\subsection*{Benefits}
The biggest benefit will be the quality of life improvement for patients who
are suffering from ALS, specifically, those that are totally paralyzed and
cannot even move their eyes. The device we are developing software for is
much cheaper than other devices on the market which are prohibitively
expensive for ALS patients, costing upwards of \$17,000. Since this is open
source, there is a large potential for this project to grow and be built
upon, contributing to future projects.

\newpage
\section*{\centering The Team}

\subsection*{Team Name}

% TODO

\subsection*{Team Members and Their Specialties}

Samuel Bever enjoys working in algorithms, artificial intelligence, and
graphics. Role: Tester.

Michael Conway has experience and interests in the areas of algorithms,
programming languages and computer security. Role: Technical Lead.

Joseph Muoio has experience and interests in the areas of AI, mobile app
development and software development. Role: Team Leader.

Kyle Patron is interested in mathematics, particularly as it relates to
signal processing and machine control. Role: Tools lead.

Kevin Zakszewski has experience in UI/UX design and front end
implementation. Role: Lead Designer.

\subsection*{Team Communication}

Our team will be meeting at least once a week to discuss our progress,
outline our tasks for that week, and address any other issues. If necessary,
more meetings will be scheduled. Most group communication outside of these
meetings will be done via email and Google Hangouts. For source code
management, we will be using a GitHub repository, which comes with an issue
tracker and a project wiki. We will also be using review board in order to
organize and distribute code reviews to the team. 

\subsection*{Team Issues}

We are all busy and have many other school and work commitments. Finding
time to meet on a regular basis will be important but possibly difficult.

There are some differences of opinion when it comes to technical aspects of
this project. We are still debating which language we should use.

\newpage
\section*{\centering The Plan}

\subsection*{Objectives}

Our goals for this first cycle are meant to begin each of the portions of
the project so that we can use the other terms to integrate our different
subsystems and poslish the resulting amalgamation.

\begin{itemize}
    \item Begin experimenting with the Emotiv system to determine its
capabilities and limitations
    \item Gather requirements from our stakeholder
    \item Construct a rough UI that will serve as a baseline for future
iterations
    \item Combine these initial elements to create a rough prototype, which
will serve to
        \begin{itemize}
            \item test the feasibility of the design
            \item identify the most important challenges to focus on in the
                remaining cycles
        \end{itemize}
    \item Present the prototype to prospective users in order to clearly
        define assumed, unstated or non-obvious requirements
\end{itemize}

\subsection*{Schedule}

\begin{tabular}{| l | l | l |}
    \hline
    Week & Person & Contribution \\
    \hline \hline

    2-3 & Everyone & Write launch report \\ \hline
    3-5 & Joe Muoio & Set up coding infrastructure \\ \hline
    3-5 & Kyle Patron & Gather requirements and meet with stakeholder \\ \hline
    3-7 & Sam Bever & Experiment with default Emotiv functionality \\ \hline
    4-8 & Kevin Zakszewski & Develop prototype UI \\ \hline
		5-8 & Kyle Patron & Develop initial system design \\ \hline
    7-10 & Mike Conway & Combine parts into prototype \\ \hline
    8-11 & Everyone & Prepare presentation \\ \hline
\end{tabular}

\begin{ganttchart}{1}{11}
    \gantttitle{Fall Term}{11} \\
    \gantttitlelist{1,...,11}{1} \\

    % These are all "at the end of week #"
    \ganttbar{Write launch report}{2}{2} \\
    \ganttlinkedmilestone{Launch Report 1}{2} \\
    \ganttbar{Set up coding infrastructure}{3}{4} \\
    \ganttbar{Gather requirements and meet with stakeholder}{3}{5} \\
    \ganttbar{Experiment with default Emotiv functionality}{3}{6} \\
    \ganttlinkedmilestone{SRS 1}{5} \\
    \ganttbar{Develop prototype UI}{4}{7} \\
    \ganttbar{Develop initial system design}{5}{7} \\
    \ganttlinkedmilestone{SDS 1}{7} \\
    \ganttbar{Combine parts into prototype}{7}{9} \\
    \ganttlinkedmilestone{Prototype 1}{9} \\
    \ganttbar{Prepare presentation}{8}{10} \\
    \ganttlinkedmilestone{Presentation 1}{10}

\end{ganttchart}

\newpage
\section*{\centering Table of Contributions}
\begin{tabular}{| l | l | l | l |}
    \hline
     & Section & Writing & Editing \\
    \hline \hline
    1 & Project & Joe Muoio  & Mike Conway \\ \hline
    2 & Team & Everyone & Sam Bever \\ \hline
    3 & Plan & Kyle Patron & Kevin Zakszewski \\ \hline
\end{tabular}
\begin{thebibliography}{1}

    \bibitem{ALSsource} The ALS Association. (2014, October, 5). \textit{What
        is ALS?} [Online]. Available:
        \url{http://www.alsa.org/about-als/what-is-als.html}

\end{thebibliography}
\newpage
\noindent I certify that:
\begin{itemize}
\item This paper/project/exam is entirely my own work.
\item I have not quoted the words of any other person from a printed source or a website without indicating what has been quoted and providing an appropriate citation.
\item I have not submitted this paper / project to satisfy the requirements of any other course.
\end{itemize}

\vspace{1cm}
\noindent\makebox[\textwidth][l]{
Signature:
\makebox[5cm][l] {\hrulefill} 
\ \ Date:
\makebox[4cm][l] {\hrulefill} 
}


\vspace{0.5cm}
\noindent\makebox[\textwidth][l]{
Signature:
\makebox[5cm][l] {\hrulefill} 
\ \ Date:
\makebox[4cm][l] {\hrulefill} 
}

\vspace{0.5cm}
\noindent\makebox[\textwidth][l]{
Signature:
\makebox[5cm][l] {\hrulefill} 
\ \ Date:
\makebox[4cm][l] {\hrulefill} 
}

\vspace{0.5cm}
\noindent\makebox[\textwidth][l]{
Signature:
\makebox[5cm][l] {\hrulefill} 
\ \ Date:
\makebox[4cm][l] {\hrulefill} 
}

\vspace{0.5cm}
\noindent\makebox[\textwidth][l]{
Signature:
\makebox[5cm][l] {\hrulefill} 
\ \ Date:
\makebox[4cm][l] {\hrulefill} 
}
\vspace{\fill}
\subsection*{Grading}
The grade is given on the basis of quality, clarity, presentation, completeness, and writing of each section in the report. This is the grade of the group. Individual grades will be assigned at the end of the term when peer reviews are collected.
\end{document}
