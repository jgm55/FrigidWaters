\documentclass{report}

\usepackage[margin=1in]{geometry}
\usepackage{datetime}
\usepackage{pgfgantt}

\newdateformat{required}{\twodigit{\THEMONTH}/\twodigit{\THEDAY}/\THEYEAR}

\begin{document}

\begin{titlepage}
    \begin{center}
        \begin{huge}
        ALS Project \\[1cm]
        Team Doocy \\[2.2cm]
        { \bfseries Launch Report } \\[1cm]
        Cycle \# 1\\[2.2cm]
        Date: \required\today\\[1cm]
        \end{huge}
    \end{center}
    \null \vfill
    \begin{large}
        Team Members: \\[0.5cm]
        Name: Samuel Bever\\[0.5cm]
        Name: Michael Conway\\[0.5cm]
        Name: Joseph Muoio\\[0.5cm]
        Name: Kyle Patron\\[0.5cm]
        Name: Kevin Zakszewski
    \end{large}
\end{titlepage}

\section*{\centering The System/Product}

\subsection*{System/Product Name}

% TODO

\subsection*{Introduction}

% Describe ALS
% Patients with late ALS can't move, so brain interface would help. Want to
% make a system to connect BCI to home automation and social media.
Amyotrophic lateral sclerosis (ALS) (also called ``Lou Gehrig's Disease'' is a neurodegenerative disease affecting the brain and spinal cord's motor nuerons. The motor nuerons progressively degenerate and eventually die off leading to the loss of muscle control. Eventually, this leads to full paralysis. Since the disease culminates in paralysis, affected individuals lose the ability to interact in the world. Even something as simple as turning on a light switch becomes impossible.


% Existing things:
% Last year's work???
% Bunch of research
% At least one thing that sounds rather similar (Thought-Wired)

he goal of the project is to build a system which people afflicted with ALS can use to interact with their house through Home Automation Systems (HAS) . This includes controlling the thermostat, lights, stereo, and television as well as interfacing with social media. By having a Brain Computer Interface (BCI) to their automated house, individuals with ALS can still interact with their environment despite full paralysis. This will build off of a project from last year. The interface will be improved and new devices will be experimented with.  

\subsection*{Highlighted Features}

\begin{itemize}
\item Allow users with ALS to perform basic household tasks without needing to move
\item Integrate different BCI devices to different HAS
\end{itemize}

\subsection*{Sponsor or Proxy User}

Our outside stakeholder is \textbf{Sara Feldman}, PT, DPT, ATP, MDA/ALS Center
of Hope, Drexel University College of Medicine.

% We will be meeting with Dr. Feldman to...

\subsection*{Issues}

% Initial impression: unclear what they want (requirements gathering tricky)
The initial impression is it will be difficult to gather requirements from the external stakeholder. There seems to not be a real plan as to what exactly they want. 
% BCIs not yet precise
BCIs are probably not the most precise devices. They also need to be trained for a particular user. Making this process as easy as possible and as accurate as possible will be a big challenge.

% Different ones out there (confirm? at least different levels of
% invasiveness); design to handle all of them with appropriate module?
Another challenge will be integrating different BCI devices with different types of home control systems. There are multiple different BCI devices and multiple different home automation systems. Our software design has to be extremely flexible in order to handle the cases of different BCI devices integrating with different home control systems. 

Investigating all the HAS and BCI systems as part of the requirements gathering will be a very important task to see everything that is out there on the market today and the assumptions those products make about their users's limitations.
% Different home control systems (ditto???)
% Usual target audience of home control is the non-handicapped. Will we need
% to fight resulting assumptions (i.e. what's out there isn't what we'd need)?

\newpage
\section*{\centering The Team}

\subsection*{Team Name}

% TODO

\subsection*{Team Members and their specialties}

Samuel Bever

Michael Conway

Joseph Muoio

Kyle Patron

Kevin Zakszewski

% Roles:

% Team leader
% ...

\subsection*{Team Communication}
Our team will be meeting at least once a week to discuss our progress, outline our tasks for that week, and  address anything else that needs to be worked out. If necessary, more meetings will be scheduled. Most group communication will be done via email. We have also created a git repository to track our work and share the necessary files and documents.

\subsection*{Team Issues}

% Discuss scheduling issues, issues with "specialty" (too many backend, not enough front end developers etc), or other forseen issues

\newpage
\section*{\centering The Plan}

\subsection*{Objectives}
Currently, we are in the first cycle of CI491. This means that the bulk of our goals and objectives for this term will focus heavily on the preparation, documentation, and rough prototype of our project. Our first goal is to complete the launch report documentation. This will outline our plan for the project. Secondly, we will also have our SRS complete. Here, we will outline the requirements of the system that we are developing. Once we have these markers and guidelines in place, we can focus on our system design. Fianlly, by the end of the term we will have a first prototype of our system. The deliverables from all these initial goals will then be compiled into a final presentation for the term demonstrating our progress, our successes, and our failures. 

\begin{itemize}
    \item Research available technologies and choose the one(s) we will use
        \begin{itemize}
            \item Home control
            \item BCI
        \end{itemize}
    \item % TODO
\end{itemize}

\subsection*{Schedule}

\begin{tabular}{| l | l | l |}
    \hline
    Week & Person & Contribution \\
    \hline \hline
    Lorem & ipsum & dolor sit amet \\ \hline
\end{tabular}

\begin{ganttchart}{1}{11}
    \gantttitle{Fall Term}{11} \\
    \gantttitlelist{1,...,11}{1} \\

    % These are all "at the end of week #"
    \ganttbar{Write launch report}{2}{2} \\
    \ganttlinkedmilestone{Launch Report 1}{2} \\
    \ganttbar{Gather requirements}{3}{5} \\
    \ganttlinkedmilestone{SRS 1}{5} \\
    \ganttbar{Design}{5}{7} \\
    \ganttlinkedmilestone{SDS 1}{7} \\
    \ganttbar{Implement prototype}{7}{9} \\
    \ganttlinkedmilestone{Prototype 1}{9} \\
    \ganttbar{Prepare presentation}{8}{10} \\
    \ganttlinkedmilestone{Presentation 1}{10}

\end{ganttchart}

\newpage
\section*{\centering Table of Contributions}
\begin{tabular}{| l | l | l | l |}
    \hline
     & Section & Writing & Editing \\
    \hline \hline
    1 & Project & & \\ \hline
    2 & Team & & \\ \hline
    3 & Plan & & \\ \hline
\end{tabular}
\newpage
\noindent I certify that:
\begin{itemize}
\item This paper/project/exam is entirely my own work.
\item I have not quoted the words of any other person from a printed source or a website without indicating what has been quoted and providing an appropriate citation.
\item I have not submitted this paper / project to satisfy the requirements of any other course.
\end{itemize}

\vspace{1cm}
\noindent\makebox[\textwidth][l]{
Signature:
\makebox[5cm][l] {\hrulefill} 
\ \ Date:
\makebox[4cm][l] {\hrulefill} 
}


\vspace{0.5cm}
\noindent\makebox[\textwidth][l]{
Signature:
\makebox[5cm][l] {\hrulefill} 
\ \ Date:
\makebox[4cm][l] {\hrulefill} 
}

\vspace{0.5cm}
\noindent\makebox[\textwidth][l]{
Signature:
\makebox[5cm][l] {\hrulefill} 
\ \ Date:
\makebox[4cm][l] {\hrulefill} 
}

\vspace{0.5cm}
\noindent\makebox[\textwidth][l]{
Signature:
\makebox[5cm][l] {\hrulefill} 
\ \ Date:
\makebox[4cm][l] {\hrulefill} 
}

\vspace{0.5cm}
\noindent\makebox[\textwidth][l]{
Signature:
\makebox[5cm][l] {\hrulefill} 
\ \ Date:
\makebox[4cm][l] {\hrulefill} 
}
\vspace{\fill}
\subsection*{Grading}
The grade is given on the basis of quality, clarity, presentation, completeness, and writing of each section in the report. This is the grade of the group. Individual grades will be assigned at the end of the term when peer reviews are collected.
\end{document}
