\documentclass{article}

\usepackage[margin=1in]{geometry}
\usepackage{parskip}
\usepackage{datetime}
\usepackage{url}
\usepackage{hyperref}
\usepackage[utf8]{inputenc}
\usepackage{graphicx}

\newdateformat{required}{\twodigit{\THEMONTH}/\twodigit{\THEDAY}/\THEYEAR}
\raggedbottom
\begin{document}

\begin{titlepage}
    \begin{center}
        \begin{huge}
        House in Your Head \\[1cm]
        Team G1-FrigidWaters \\[2.2cm]
        { \bfseries System Design Specification } \\[1cm]
        Cycle \# 1\\[2.2cm]
        Date: \required\today\\[1cm]
        \end{huge}
    \end{center}
    \null \vfill
    \begin{large}
        Team Members: \\[0.5cm]
        Name: Samuel Bever\\[0.5cm]
        Name: Michael Conway\\[0.5cm]
        Name: Joseph Muoio\\[0.5cm]
        Name: Kyle Patron\\[0.5cm]
        Name: Kevin Zakszewski
    \end{large}
\end{titlepage}
\section*{\centering Table of Contents}
\makeatletter
\@starttoc{toc}
\newcommand{\hsubsubsection}{
\@startsection{subsubsection}{3}{\z@}%
                                     {-3.25ex\@plus -1ex \@minus -.2ex}%
                                     {-1.5ex \@plus -.2ex}% Formerly 1.5ex \@plus .2ex
                                     {R\normalfont\normalsize}}
\newcommand{\hparagraph}{
\@startsection{paragraph}{4}{\z@}%
                                     {-3.25ex\@plus -1ex \@minus -.2ex}%
                                     {-1.5ex \@plus -.2ex}% Formerly 1.5ex \@plus .2ex
                                     {R\normalfont\normalsize}}
\newcommand{\hsubparagraph}{
\@startsection{subparagraph}{5}{\z@}%
                                     {-3.25ex\@plus -1ex \@minus -.2ex}%
                                     {-1.5ex \@plus -.2ex}% Formerly 1.5ex \@plus .2ex
                                     {R\normalfont\normalsize}}
\setcounter{secnumdepth}{5}
\makeatother
\newpage
 

\section{Introduction}

[The structure of this introduction is similar to the introduction section of SRS.]

\subsection{Purpose}

[This subsection talks about the purpose of this SDS report not about the purpose of the project.]
The purpose of this document is to describe

\subsection{Scope}

[This subsection describes a short overview of the system.  Defining the scope of the project and identify aspects that are NOT included in the system.  Make it clear to the readers of the report who are the developers.  Write it from the system perspective (not from the user perspective in SRS).]

\subsection{Definitions, Acronyms, and Abbreviations}

[If there are definitions, acronyms, and abbreviation that you are using frequently in the report, define them here in this subsection so that the readers can always refer to here.]

Same??

\begin{description}
    \item[EEG] Can refer to:
        \begin{itemize}
            \item Electroencephalography - Recording of the brain's electrical
                activity 
	        \item Electroencephalogram - The device that is used to record the
	            brain's electrical activity
        \end{itemize}
    \item[Emotiv] The electroencephalogram hardware device, created by Emotiv
        Limited, used to read the user's brain activity (EEG)
    \item[Brain Computer Interface (BCI)] The class of devices that the Emotiv
        belongs to
    \item[Amyotrophic Lateral Sclerosis (ALS)] A neurodegenerative disorder
        that our target users suffer from. The main characteristics of ALS
        that we are concerned with in the scope of this project are the
        limited movement and mobility to complete paralysis.
\end{description}


\subsection{Overview}


[The section is describing the rest of the SDS document, providing a roadmap of what are going to be covered in the next few sections.]
Section 2 discusses ...

\newpage

\section{Requirements Traceability Matrix}
This section maps the relationship between requirement statements and detailed design entities.  As such it shows how requirements are covered by the design, and demonstrates the purpose for which design entity exists.
The values in the cells of the table show which requirements provide the purpose for each entity. The cell values are:

Blank – the design entity does not implement any of that requirement
P for Primary - the design entity implements all or most of the requirement
S for Secondary – the design entity implements a smaller but essential part of the requirement

\section{Architectural Description}
[In IEEE 1016, this section is called Decomposition Description.  You should provide your DFD decomposition here.] 

\subsection{External Interfaces}

This section gives a description of the hardware and software interfaces. Also included is a basic prototype of the UI.

\subsection{Overview of modules / components}

[Present your DFDs and describe each DFD you present here.]

\subsection{Structure and relationships}
[Present the details of each component in your DFDs.  The IDs should match to those in DFDs.]

\subsubsection{Component <ID 1>}

\subsubsection{Component <ID 1>}
.... etc

\section{Interface Description}

\subsection{User Interface}

[Describe the detailed design of user interface.  Include diagram to show screens and how they relate to each other.]

\subsection{Data Interface}
[Define data flowing in and out of your system.  Include the ERD here.]

\subsection{Programming Interface} 
[If your system allows other system to interact with your components, define the programming interfaces here.]

\newpage

\section{Detailed Design}

\subsection{Component Template Description}

\subsection{<Entity ID> - <Entity Name>}
 [Read IEEE 1016 Sections 5.2 and 5.3 about design entities and design entity attributes.  
A design entity is an element of a design that is structurally and functionally distinct from othere elements and that is separately named and referenced.  Entities can be a system, subsystems, data stores, modules, programs, and processes.  Each entity has a set of entity attributes including, name, identification, type, purpose, function, subordinates, dependencies, interface, resources, processing, and data.]

\textbf{Type} [the kind of entity]
\textbf{Purpose} [why the entity exist?]
\textbf{Function} [a statement of what the entity does.]
\textbf{Subordinates} [the identification of all entities composing this entity]
\textbf{Dependencies} [the relationship of this entity with other entities]
\textbf{Interface} [how this entity interact with other entities]
\textbf{Resources} [the description of the elements used by this entity that are external to the system]
\textbf{Processing} [the algorithm used by this entity to perform its function]
\textbf{Data} [the data elements, should be the same those in data dictionary]

\newpage

\section{Reuse and Relationships to other Products [optional]}

\newpage

\section{Design decisions and tradeoffs [optional]}

\newpage

\section{Psuedocode for components}

\newpage

\section{Appendicies}

\newpage

\section*{\centering Table of Contributions}
\begin{tabular}{| l | l | l | l |}
    \hline
     & Section & Writing & Editing \\
    \hline \hline
		1 & 2.2, 2.6, 3.3, 3.4, 3.6 & Kyle Patron & Michael Conway\\ \hline
		2 & 1, 2.1, 3.1 & Joe Muoio & Michael Conway\\ \hline
		3 & 2.4, 2.5 & Sam Bever & Michael Conway\\ \hline
		4 & 3.2 & Michael Conway & Kevin Zakszewski \\ \hline
		5 & 1, 2.3 & Kevin Zakszewski & Michael Conway \\ \hline
\end{tabular}
\newpage
\noindent I certify that:
\begin{itemize}
\item This paper/project/exam is entirely my own work.
\item I have not quoted the words of any other person from a printed source or a website without indicating what has been quoted and providing an appropriate citation.
\item I have not submitted this paper / project to satisfy the requirements of any other course.
\end{itemize}

\vspace{1cm}
\noindent\makebox[\textwidth][l]{
Signature:
\makebox[5cm][l] {\underline{Samuel Bever}} 
\ \ Date:
\makebox[4cm][l] {\underline{\required\today}} 
}


\vspace{0.5cm}
\noindent\makebox[\textwidth][l]{
Signature:
\makebox[5cm][l] {\underline{Michael Conway}} 
\ \ Date:
\makebox[4cm][l] {\underline{\required\today}} 
}

\vspace{0.5cm}
\noindent\makebox[\textwidth][l]{
Signature:
\makebox[5cm][l] {\underline{Joe Muoio}} 
\ \ Date:
\makebox[4cm][l] {\underline{\required\today}} 
}

\vspace{0.5cm}
\noindent\makebox[\textwidth][l]{
Signature:
\makebox[5cm][l] {\underline{Kyle Patron}} 
\ \ Date:
\makebox[4cm][l] {\underline{\required\today}} 
}

\vspace{0.5cm}
\noindent\makebox[\textwidth][l]{
Signature:
\makebox[5cm][l] {\underline{Kevin Zakszewski}} 
\ \ Date:
\makebox[4cm][l] {\underline{\required\today}} 
}

\vspace{\fill}
\subsection*{Grading}
The grade is given on the basis of quality, clarity, presentation, completeness, and writing of each section in the report. This is the grade of the group. Individual grades will be assigned at the end of the term when peer reviews are collected.
\end{document}
