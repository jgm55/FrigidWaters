\documentclass{article}

\usepackage[margin=1in]{geometry}
\usepackage{parskip}
\usepackage{datetime}
\usepackage{url}
\usepackage{hyperref}
\usepackage[utf8]{inputenc}
\usepackage{graphicx}
\usepackage{listings}

\lstset{
    basicstyle=\ttfamily
}

\newdateformat{required}{\twodigit{\THEMONTH}/\twodigit{\THEDAY}/\THEYEAR}
\raggedbottom

\begin{document}

\begin{titlepage}
    \begin{center}
        \begin{huge}
        House in Your Head \\[1cm]
        Team G1-FrigidWaters \\[2.2cm]
        { \bfseries System Design Specification } \\[1cm]
        Cycle \# 1\\[2.2cm]
        Date: \required\today\\[1cm]
        \end{huge}
    \end{center}
    \null \vfill
    \begin{large}
        Team Members: \\[0.5cm]
        Name: Samuel Bever\\[0.5cm]
        Name: Michael Conway\\[0.5cm]
        Name: Joseph Muoio\\[0.5cm]
        Name: Kyle Patron\\[0.5cm]
        Name: Kevin Zakszewski
    \end{large}
\end{titlepage}
\section*{\centering Table of Contents}
\makeatletter
\@starttoc{toc}
\newcommand{\hsubsubsection}{
\@startsection{subsubsection}{3}{\z@}%
                                     {-3.25ex\@plus -1ex \@minus -.2ex}%
                                     {-1.5ex \@plus -.2ex}% Formerly 1.5ex \@plus .2ex
                                     {R\normalfont\normalsize}}
\newcommand{\hparagraph}{
\@startsection{paragraph}{4}{\z@}%
                                     {-3.25ex\@plus -1ex \@minus -.2ex}%
                                     {-1.5ex \@plus -.2ex}% Formerly 1.5ex \@plus .2ex
                                     {R\normalfont\normalsize}}
\newcommand{\hsubparagraph}{
\@startsection{subparagraph}{5}{\z@}%
                                     {-3.25ex\@plus -1ex \@minus -.2ex}%
                                     {-1.5ex \@plus -.2ex}% Formerly 1.5ex \@plus .2ex
                                     {R\normalfont\normalsize}}
\setcounter{secnumdepth}{5}
\makeatother
\newpage
 

\section{Introduction - Kevin all unless otherwise marked}

[The structure of this introduction is similar to the introduction section of SRS.]

\subsection{Purpose}

[This subsection talks about the purpose of this SDS report not about the purpose of the project.]
The purpose of this document is to describe the data, interface, architectural, and component-level design for our project. The architectural section will go over the modules and components, their relationship, and will describe the structure of the system as a whole. The interface section contains more detail on the system's user interface, data interface,  and the programming interface. Finally, in the remaining portion of this document, any other design details or relationships will be layed out .

\subsection{Scope}

[This subsection describes a short overview of the system.  Defining the scope of the project and identify aspects that are NOT included in the system.  Make it clear to the readers of the report who are the developers.  Write it from the system perspective (not from the user perspective in SRS).]

This system is intended to be a Brain Computer Interface that will act as a facade between data received from the Emotiv device and a home automation system. The system on the end of the Emotiv device and the home automation system are a a bit like black boxes. We are more concerned with taking the data, interpreting it correctly and triggering the correct actions in the home automation system. The functionality of the other two systems are out of our scope. 

\subsection{Definitions, Acronyms, and Abbreviations}

[If there are definitions, acronyms, and abbreviation that you are using frequently in the report, define them here in this subsection so that the readers can always refer to here.]

Same??

\begin{description}
    \item[EEG] Can refer to:
        \begin{itemize}
            \item Electroencephalography - Recording of the brain's electrical
                activity 
	        \item Electroencephalogram - The device that is used to record the
	            brain's electrical activity
        \end{itemize}
    \item[Emotiv] The electroencephalogram hardware device, created by Emotiv
        Limited, used to read the user's brain activity (EEG)
    \item[Brain Computer Interface (BCI)] The class of devices that the Emotiv
        belongs to
    \item[Amyotrophic Lateral Sclerosis (ALS)] A neurodegenerative disorder
        that our target users suffer from. The main characteristics of ALS
        that we are concerned with in the scope of this project are the
        limited movement and mobility to complete paralysis.
    \item[API] software level functions which allow other parts of the system (or external systems) to interact and share data.
\end{description}


\subsection{Overview}


[The section is describing the rest of the SDS document, providing a roadmap of what are going to be covered in the next few sections.]
Section 2 discusses ...

\subsection{Requirements Traceability Matrix - Joe}
This section maps the relationship between requirement statements and detailed design entities.  As such it shows how requirements are covered by the design, and demonstrates the purpose for which design entity exists.
The values in the cells of the table show which requirements provide the purpose for each entity. The cell values are:

Blank – the design entity does not implement any of that requirement
P for Primary - the design entity implements all or most of the requirement
S for Secondary – the design entity implements a smaller but essential part of the requirement

\newpage

\section{Architectural Description - kyle}
[In IEEE 1016, this section is called Decomposition Description.  You should provide your DFD decomposition here.] 

\subsection{External Interfaces}

This section gives a description of the hardware and software interfaces. Also included is a basic prototype of the UI.

\subsection{Overview of modules / components}

[Present your DFDs and describe each DFD you present here.]

\subsection{Structure and relationships}
[Present the details of each component in your DFDs.  The IDs should match to those in DFDs.]

\subsubsection{Component <ID 1>}

\subsubsection{Component <ID 1>}
.... etc

\newpage

\section{Interface Description}

\subsection{User Interface - Kevin}

The user interface of the system is a rather simple one because most of the data we will receive and the choices the user will make are binary. There will be one type of screen where the user will cycle through the options of the objects that they can interact with in the home automation system. This is also the screen where the user makes the decision which object they want to interact with. When an object is chosen, they will be directed to the second type of screen. The second type of screen that handles the changing of the state of the object the user selects. From this screen the user can either change the state or return to the previous screen (the first type where the users cycle through the options). These two types UI of screen will make up the majority of the system. These designs will be modified or extra user interface designs will be implemented as necessary. 

\subsection{Data Interface - Kyle}
[Define data flowing in and out of your system.  Include the ERD here.]

\subsection{Programming Interface} 
There is no need for a programming interface because it is a generic, self-contained system.

\newpage

\section{Detailed Design}

\subsection{Component Template Description - Joe}

\subsection{Design Entities}

\subsubsection*{1.0 - EEG Interpreter}
\begin{tabular}{ | l |  p{13.3cm} |}
\hline
\textbf{Type} & Module \\ \hline
\textbf{Purpose} & Determines if the current brain scan information matches the trained data \\ \hline
\textbf{Function} & Uses statistical techniques to look at the current EEG information coming from the user's brainwaves and compares this to the data that was trained to see if it matches one of the trained classes. \\ \hline
\textbf{Subordinates} & None \\ \hline
\textbf{Dependencies} & Relies on Training datastore (2.2). There needs to be trained models to compare the stream of data to. \\ \hline
\textbf{Interface} & The EEG Interpreter will have an API that can be polled by other components. \\ \hline
\textbf{Resources} & The EEG information comes from the emotive device. \\ \hline
\textbf{Processing} & Various statistical techniques will be explored. At first, a simple algorithm that checks whether the EEG information falls within a given threshold to the training data, but more advanced techniques will be explored as well. \\ \hline
\textbf{Data} & See ERD in Section 3 \\ \hline
\end{tabular}

\subsubsection*{2.0 - Training Module}
\begin{tabular}{ | l |  p{13.3cm} |}
\hline
\textbf{Type} & Module \\ \hline
\textbf{Purpose} & Trains a model for the current user for each state. \\ \hline
\textbf{Function} & Delegates to state trainers for each state that needs to be trained for the current user. Then stores this in the Training Datastore (2.2). \\ \hline
\textbf{Subordinates} & State Trainer (2.1). Every state that needs to be trained will be trained through a specific state trainer for that state. \\ \hline
\textbf{Dependencies} & Training Datastore (2.2). This is needed to save the trained models. \\ \hline
\textbf{Interface} & None \\ \hline
\textbf{Resources} & None \\ \hline
\textbf{Processing} & For each State Trainer (2.1) that is untrained, train the state model for that trainer. Then, save the results to the Training Datastore (2.2). \\ \hline
\textbf{Data} & See ERD in Section 3 \\ \hline
\end{tabular}

\subsubsection*{2.1 - State Trainer}
\begin{tabular}{ | l |  p{13.3cm} |}
\hline
\textbf{Type} & Module \\ \hline
\textbf{Purpose} & Trains a model for a single state. \\ \hline
\textbf{Function} & Trains a model for a specific state. Each state trainer trains on one state, so there will be a State Trainer for the resting state, active state, and any other states deemed necessary in the future. \\ \hline
\textbf{Subordinates} & None \\ \hline
\textbf{Dependencies} & GUI Module (3.0). To train a model, interaction with the user of the system is required. This is done through a Graphic User Interface.  \\ \hline
\textbf{Interface} & None \\ \hline
\textbf{Resources} & None \\ \hline
\textbf{Processing} & Interact with the user on the screen until a model is built based on the user's personal EEG. \\ \hline
\textbf{Data} & See ERD in Section 3 \\ \hline
\end{tabular}

\subsubsection*{2.2 - Training Datastore}
\begin{tabular}{ | l |  p{13.3cm} |}
\hline
\textbf{Type} & Data store \\ \hline
\textbf{Purpose} & Holds the trained data for the current user. \\ \hline
\textbf{Function} & Holds all the trained states for the current user of the system. This includes the resting state and the active state at a minimum. Other states will be explored in the future. \\ \hline
\textbf{Subordinates} & None \\ \hline
\textbf{Dependencies} & None \\ \hline
\textbf{Interface} & The training datastore will have an API that allows read and write access to it. \\ \hline
\textbf{Resources} & May need hard disk space if it will be saved permanently.  \\ \hline
\textbf{Processing} & The currently loaded user's data will be the only data retrieved. Loading from disk and saving to disk are allowed. \\ \hline
\textbf{Data} & See ERD in Section 3 \\ \hline
\end{tabular}

\subsubsection*{3.0 - GUI Module}
\begin{tabular}{ | l |  p{13.3cm} |}
\hline
\textbf{Type} & Module \\ \hline
\textbf{Purpose} & Hands output from the system to an external monitor. \\ \hline
\textbf{Function} & All interaction with the user is through an external monitor. This system handles all output to the screen. \\ \hline
\textbf{Subordinates} & None \\ \hline
\textbf{Dependencies} & None \\ \hline
\textbf{Interface} & There will be a writer API to allow other parts of the system to output to the screen. \\ \hline
\textbf{Resources} & External Monitor \\ \hline
\textbf{Processing} & The GUI module will wait until given something to write to the screen. When it is given a request, it will write that request to the screen. \\ \hline
\textbf{Data} & See ERD in Section 3 \\ \hline
\end{tabular}

\subsection*{4.0 - Home Automation Interface}
\begin{tabular}{ | l |  p{13.3cm} |}
\hline
\textbf{Type} & Module \\ \hline
\textbf{Purpose} & Interfaces with external home automation controls. \\ \hline
\textbf{Function} & Manages the system's interface with external home
automation controls. This includes providing the rest of the system with
information about available controls, maintaining any necessary state, and
processing requests to manipulate the home via the controls. \\ \hline
\textbf{Subordinates} & None \\ \hline
\textbf{Dependencies} & None \\ \hline
\textbf{Interface} & The interface will support querying for information about
available home controls and to manipulate the home via the controls. \\ \hline
\textbf{Resources} & Home automation system controls \\ \hline
\textbf{Processing} & The module will query control state as necessary and
reject requests that are not valid in the current state. \\ \hline
\textbf{Data} & See ERD in Section 3 \\ \hline
\end{tabular}

\newpage

\section{Reuse and Relationships to other Products [optional] - Sam}

\newpage

\section{Design decisions and tradeoffs [optional] - Sam}
why pick emotiv
why binary
why house
why brain

\newpage

\section{Psuedocode for components - Mike}

\subsection*{1.0 - EEG Interpreter}

% TODO Does this make any sense?

\begin{lstlisting}
repeat forever:
    monitor EEG
    if EEG in a trained state:
        notify all registered listeners
        % TODO OR call all registered callbacks
\end{lstlisting}

\subsection*{2.0 - Training Module}

Upon training request:

\begin{lstlisting}
for each untrained State Trainer:
    run the trainer
save results to Training Datastore
\end{lstlisting}

Each State Trainer:

\begin{lstlisting}
repeat until trained:
    display prompt
    collect data from EEG Interpreter
    update training parameters
\end{lstlisting}

\subsection*{3.0 - GUI Module}

Upon menu display request:
\begin{lstlisting}
% TODO OPTION A
display menu
for each menu option:
    register callback for corresponding state in EEG Interpreter
% TODO OPTION B
display menu
register self as EEG Interpreter listener for necessary states
upon state notification, perform action
\end{lstlisting}

\subsection*{4.0 - Home Automation Interface}

Upon receiving a control request:

\begin{lstlisting}
if request is valid:
    perform action
else:
    report error
\end{lstlisting}

\newpage

\section{Appendicies - Mike}

% N/A?

\newpage

\section*{\centering Table of Contributions}
\begin{tabular}{| l | l | l | l |}
    \hline
     & Section & Writing & Editing \\
    \hline \hline
		1 & 2.2, 2.6, 3.3, 3.4, 3.6 & Kyle Patron & Michael Conway\\ \hline
		2 & 1, 2.1, 3.1 & Joe Muoio & Michael Conway\\ \hline
		3 & 2.4, 2.5 & Sam Bever & Michael Conway\\ \hline
		4 & 3.2 & Michael Conway & Kevin Zakszewski \\ \hline
		5 & 1, 2.3 & Kevin Zakszewski & Michael Conway \\ \hline
\end{tabular}
\newpage
\noindent I certify that:
\begin{itemize}
\item This paper/project/exam is entirely my own work.
\item I have not quoted the words of any other person from a printed source or a website without indicating what has been quoted and providing an appropriate citation.
\item I have not submitted this paper / project to satisfy the requirements of any other course.
\end{itemize}

\vspace{1cm}
\noindent\makebox[\textwidth][l]{
Signature:
\makebox[5cm][l] {\underline{Samuel Bever}} 
\ \ Date:
\makebox[4cm][l] {\underline{\required\today}} 
}


\vspace{0.5cm}
\noindent\makebox[\textwidth][l]{
Signature:
\makebox[5cm][l] {\underline{Michael Conway}} 
\ \ Date:
\makebox[4cm][l] {\underline{\required\today}} 
}

\vspace{0.5cm}
\noindent\makebox[\textwidth][l]{
Signature:
\makebox[5cm][l] {\underline{Joe Muoio}} 
\ \ Date:
\makebox[4cm][l] {\underline{\required\today}} 
}

\vspace{0.5cm}
\noindent\makebox[\textwidth][l]{
Signature:
\makebox[5cm][l] {\underline{Kyle Patron}} 
\ \ Date:
\makebox[4cm][l] {\underline{\required\today}} 
}

\vspace{0.5cm}
\noindent\makebox[\textwidth][l]{
Signature:
\makebox[5cm][l] {\underline{Kevin Zakszewski}} 
\ \ Date:
\makebox[4cm][l] {\underline{\required\today}} 
}

\vspace{\fill}
\subsection*{Grading}
The grade is given on the basis of quality, clarity, presentation, completeness, and writing of each section in the report. This is the grade of the group. Individual grades will be assigned at the end of the term when peer reviews are collected.
\end{document}
